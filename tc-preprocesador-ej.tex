
\section{Ejercicios}
\label{sec:tc-preprocesador-ej}
\begin{enumerate}
	\item Dé ejemplos de directivas de preprocesador:
		\begin{enumerate}[label=\alph*.]
\item Para incluir un archivo proporcionado por el compilador.
\item Para incluir un archivo confeccionado por el usuario.
\item Para definir una constante numérica.
\item Para compilar un segmento de programa bajo la condición de estar definida una constante.
\item Idem bajo la condición de ser no definida.
\item Idem bajo la condición de que un símbolo valga una cierta constante.
\item Idem bajo la condición de que dos símbolos sean equivalentes.
\end{enumerate}
\item Proponga un método para incluir un conjunto de archivos en un módulo fuente con una sola
directiva de preprocesador.
 \item ¿Cuál es el ámbito de una definición de preprocesador? Si defino un símbolo A en un fuente y lo
compilo creando un módulo objeto algo.o, ¿puedo utilizar A desde otro fuente, sin declararlo, a
condición de linkeditarlo con algo.o?
\item ¿Qué pasa si defino dos veces el mismo símbolo en un mismo fuente?
\item Un cierto header A es incluido por otros headers B, C y D. El fuente E necesita incluir a B, C y D.
Proponga un método para poder hacerlo sin obligar al preprocesador a leer el header A más de una
vez.
\item Edite el programa hello.c del ejemplo del capítulo 1 reemplazando la cadena \lstinline{"Hola, mundo!\n"} por
un símbolo definido a nivel de preprocesador.
\item  Edite el programa hello.c incluyendo la compilación condicional de la instrucción de impresión
printf() sujeta a que esté definido un símbolo de preprocesador llamado IMPRIMIR. Compile y pruebe
a) sin definir el símbolo IMPRIMIR, b) definiéndolo con una directiva de preprocesador, c)
definiéndolo con una opción del compilador. ¿En qué casos es necesario recompilar el programa?
\item  Escriba una macro que imprima su argumento usando la función printf(). Aplíquela para reescribir
hello.c de modo que funcione igual que antes.
\item  ¿Cuál es el resultado de preprocesar las líneas que siguen? Es decir, ¿qué recibe exactamente el
compilador luego del preprocesado?
\begin{lstlisting}
#define ALFA 8
#define BETA 2*ALFA
#define PROMEDIO(x,y) (x+y)/2
a=ALFA*BETA;
b=5;
c=PROMEDIO(a,b);
\end{lstlisting}
\item  ¿Qué está mal en los ejemplos que siguen?
	\begin{enumerate}[label=\alph*.]

\item  
\begin{lstlisting}
#define PRECIO 27.5
PRECIO=27.7;
\end{lstlisting}

\item  
\begin{lstlisting}
#define 3.14 PI
\end{lstlisting}

\item  
\begin{lstlisting}
#define doble(x) 2*x;
alfa=doble(6)+5;
\end{lstlisting}
\end{enumerate}
\item Investigue la función de los símbolos predefinidos \lstinline{__STDC__}, \lstinline{__FILE__} y \lstinline{__LINE__}.
\end{enumerate}


