

\chapter{Estructuras de control}

Las estructuras de control de C no presentan, en conjunto, grandes diferencias
con las del resto de los lenguajes estructurados. En los esquemas siguientes, donde figura una sentencia puede reemplazarse por
varias sentencias encerradas entre llaves (un \textbf{bloque}).

\section{Estructura alternativa}

Como en la casi totalidad de los lenguajes de programación, la estructura alternativa permite ejecutar un bloque de una o más sentencias cuando la expresión condicional es $V$, y opcionalmente otro bloque cuando dicha expresión es $F$.
Formas típicas de la estructura alternativa son las siguientes.

\begin{lstlisting}
if(expresión)
    sentencia;
\end{lstlisting}

\begin{lstlisting}
if(expresión) {
    sentencias
    ...
}
\end{lstlisting}

\begin{lstlisting}
if(expresión)
    sentencia;
else
    sentencia;
\end{lstlisting}

\begin{ejemplo}
El formato en C es libre, y las llaves sólo son necesarias cuando las sentencias de ejecución condicional son más de una. 
\begin{lstlisting}
if(a == 8) c++;
\end{lstlisting}
Las condiciones lógicas se construyen en base a operadores de relación y lógicos. 
\begin{lstlisting}
if(c >= 2 || fun(b) < 0)
\end{lstlisting}
La cláusula \lstinline{else} indica el bloque que se ejecutará en el caso negativo de la condición. 
\begin{lstlisting}
if(d)
	c++;
else
	c += 2;
\end{lstlisting}
En las estructuras anidadas, la cláusula \lstinline{else} se aparea
con el \lstinline{if} más interno, salvo que las llaves expresen otra cosa. 

En el Cuadro \ref{tab:indent}, las llaves del \textbf{Listado 1} muestran cómo están asociadas las estructuras. En el \textbf{Listado 2}, se han quitado, con lo cual la semántica cambia, pero se mantiene la misma indentación, lo que puede dar lugar a confusión. El \textbf{Listado 2} en realidad es equivalente al \textbf{Listado 3}, que utiliza llaves aunque en forma redundante, y además tiene la indentación correcta.
\end{ejemplo}

\begin{comment}
\begin{table}
\centering
\begin{tabular}{l|l|l}
Listado 1 & Listado 2 & Listado 3\\
\hline
\begin{codecell}
if(c >= 2) {
    if(d)
        c++;
} else
    c += 2;
\end{codecell}
&
\begin{codecell}
if(c >= 2) 
    if(d)
        c++;
else
    c += 2;
\end{codecell}
&
\begin{codecell}
if(c >= 2) { 
    if(d)
        c++;
	else
    	c += 2;
}
\end{codecell}\\
\end{tabular}
\caption{Estructuras alternativas anidadas e indentación.}
\label{tab:indent}
\end{table}
\end{comment}

\importante{En general, el estilo de programación debe sugerir la estructura, usando indentación; pero es, por supuesto,
la ubicación de las llaves, y no la indentación, la que determina la
estructura.}


\section{Estructuras repetitivas}

El lenguaje C dispone de las estructuras adecuadas para procesar conjuntos de \textbf{0 o más}
datos (estructura \lstinline{while}) y \textbf{1 o más} datos (estructura \lstinline{do-while}).

Ambas estructuras ejecutan su cuerpo de sentencias mientras la expresión
resulte verdadera, pero en un lazo \lstinline{while}, la comprobación de la expresión se hace \textbf{al principio de cada
ciclo}. En cambio, en el lazo \lstinline{do-while}, se hace \textbf{al final}.


\subsection{Estructura while}

\begin{lstlisting}
while(expresión)
    sentencia;
\end{lstlisting}

\subsection{Estructura do-while}

\begin{lstlisting}
do {
    sentencias;
} while(expresión);
\end{lstlisting}


\subsection{Estructura for}

La iteración es un caso particular de lazo \lstinline{while} donde necesitamos que un
bloque de sentencias se repita una cantidad previamente conocida de veces.
Estos casos implican la inicialización de variables de control, el incremento o
decremento de las mismas, y la comprobación por valor límite.
Estas tareas administrativas se pueden hacer más cómoda y expresivamente con un
lazo \lstinline{for}. 

El esquema es:
\begin{lstlisting}
for(inicialización; condición_mientras;  incremento)
    sentencia;
\end{lstlisting}

Donde:
\begin{itemize}
	\item \textbf{inicialización} es una o más sentencias, separadas por comas, que se
      ejecutan una única vez al entrar al lazo.
	\item \textbf{condición\_mientras} es una expresión lógica, que se comprueba al principio
      de cada iteración. Mientras resulte verdadera, se continúa ejecutando el
      cuerpo.
	\item \textbf{incremento} es una o más sentencias, separadas por comas, que se realizan
      al final de cada ejecución del cuerpo de la iteración. 
\end{itemize}
    
La estructura for es equivalente al siguiente lazo while:
\begin{lstlisting}
inicialización;
while(condición_mientras) {
    sentencia;
    incremento;
}
\end{lstlisting}

Aunque el uso más común de las \textbf{sentencias de incremento} es hacer avanzar o retroceder
un contador de la cantidad de iteraciones, nada impide que se utilice esa
sección para cualquier otro fin. 

Cualesquiera de las secciones inicialización, condición\_mientras o incremento
pueden estar vacías. En particular, la sentencia:
\begin{lstlisting}
for( ; ; )
\end{lstlisting}
es un lazo infinito.

\begin{ejemplo}\noindent
\begin{itemize}
	\item El siguiente lazo acumula los números 1 a 10 sobre la variable \lstinline{a}:
\begin{lstlisting}
for(i=1; i<=10; i++)
    a += i;
\end{lstlisting}

\item Si se quiere asegurar que la variable \lstinline{a} tiene un valor inicial cero, se puede
escribir:
\begin{lstlisting}
for(i=1, a=0; i<=10; i++)
	a += i;
\end{lstlisting}

\item Aprovechando la propiedad del corto circuito en las expresiones lógicas, se
puede introducir el cuerpo del lazo \lstinline{for} en la comprobación (aunque no es
recomendable si complica la lectura):

\begin{lstlisting}
for(i=1, a=0; i<=10 && a+=i; i++);
\end{lstlisting}

Nótese que el cuerpo de este último for es la sentencia nula. A propósito: es
un error muy común utilizar un signo ";" de más, así:
\begin{lstlisting}
for(i=1; i<=10; i++);
    a += i;
\end{lstlisting}

Esta estructura llevará la variable \lstinline{i} desde 1 hasta 10 sin ejecutar ningún otro
trabajo (lo que se repite es la sentencia nula) y después incrementará \lstinline{a}, una
sola vez, en el valor de la última iteración de \lstinline{i}.

\item Una clásica estructura de control para un proceso consumidor de objetos:
\begin{lstlisting}
a=leercaracter();
while( a != '\033' ) {
    procesar(a);
    a = leercaracter();
}
\end{lstlisting}

\item La propiedad de que toda asignación en C tiene un valor como expresión (el valor
asignado) permite reescribir la estructura de control anterior como:

\begin{lstlisting}
while( (a=leercaracter()) != '\033' )
        procesar(a);
\end{lstlisting}

\item Las expresiones conectadas por los operadores lógicos se evalúan de izquierda a
derecha, y la evaluación se detiene apenas alcanza a determinarse el valor de
verdad de la expresión (propiedad \quotes{del corto circuito}). Así, si suponemos que
procesar() siempre devuelve un valor distinto de cero, la sentencia siguiente equivale a los lazos anteriores.

\begin{lstlisting}
while((a=leercaracter()) != '\033' && procesar(a));
\end{lstlisting}
\item Otra versión, utilizando la estructura do-while, podría ser:
\begin{lstlisting}
do {
    if((a=leercaracter()) != '\033')
        procesar(a);
} while(a != '\033');
\end{lstlisting}

\item Si utilizamos \lstinline{for}, que es esencialmente un \textit{while}:

\begin{lstlisting}
for( ; (a=leercaracter()) != '\033'; ) procesar(a);
\end{lstlisting}

Aquí dejamos vacías las secciones de inicialización y de incremento. 
\item También, pero algo menos claro:

\begin{lstlisting}
for( ; (a=leercaracter()) != '\033'; procesar(a) );
\end{lstlisting}
\end{itemize}
\end{ejemplo}

\section{Estructura de selección}

La estructura de selección (\textit{switch}) es una estructura alternativa múltiple. Dadas varias alternativas, la estructura de selección desvía el control al
segmento de programa correspondiente. La sintaxis de la estructura \lstinline{switch} es
como sigue:
\begin{lstlisting}
switch(expresión_entera) {
    case expresión_constante1:
        sentencias;
        break;
    case expresión_constante2:
        sentencias;
        break;
    default:
        sentencias;
}
\end{lstlisting}
Al entrar al switch, se comprueba el valor de la expresión entera, y si
coincide con alguna de las constantes propuestas en los rótulos \lstinline{case}, se deriva
el control directamente allí. La sección \lstinline{default} no es obligatoria. Sirve para
derivar allí todos los casos que no se contemplen explícitamente.
En las expresiones\_constantes no se permite la aparición de variables ni
funciones. Un ejemplo válido con expresiones\_constantes sería:

\begin{lstlisting}
#define ARRIBA 10
#define ABAJO   8

switch(valor(tecla)) {
    case 127+ARRIBA:
        arriba();
        break;
    case 127+ABAJO:
        abajo();
        break;
}
\end{lstlisting}
La sentencia \lstinline{break}
 es necesaria aquí porque, al contrario que en Pascal, el
control no se detiene al llegar al siguiente rótulo.

\begin{ejemplo}
Esta estructura recibe como entrada las variables \lstinline{m} y \lstinline{a} (mes y año) y da como
salida \lstinline{d} (la cantidad de días del mes).

\begin{lstlisting}
switch(m) {
    case 2:
        d=28 + bisiesto(a) ? 1 : 0;
        break;
    case 4:
    case 6:
    case 9:
    case 11:
        d= 30;
        break; 
	default:
        d= 31;
    }
\end{lstlisting}
Si \lstinline{m} vale 4, 6, 9, u 11, la variable \lstinline{d} recibe el valor 30. Al no haber un \lstinline{break}
intermedio, el control cae hasta la asignación \lstinline{d = 30}.

La estructura \lstinline{switch} tiene algunas limitaciones con respecto a sus análogos en otros lenguajes. A saber, no se puede comparar la expresión de selección con
expresiones no constantes, ni utilizar rangos (el concepto de rango no está
definido en C).
\end{ejemplo}

\section{Transferencia incondicional}
Hay varias sentencias de transferencia incondicional de control en C. Algunas
tienen aplicación exclusivamente como modificadoras del control dentro de
estructuras, como \lstinline{break} y \lstinline{continue}.

\subsection{Sentencia continue}
Utilizada dentro de un lazo \lstinline{while}, \lstinline{do-while} o \lstinline{for}, hace que el control salte
directamente a la comprobación de la condición de iteración. Así:

\begin{lstlisting}
for(i=0; i<100; i++) {
    if(no_procesar(i))
        continue;
    procesar(i);
}
\end{lstlisting}

En este lazo, si la función \lstinline{no_procesar()} devuelve valor distinto de cero, no
se ejecuta el resto del lazo (la función \lstinline{procesar()}, y otras, si las hubiera,
hasta la llave final del lazo). Se comprueba la validez de la expresión \lstinline{i<100},
y si corresponde se inicia una nueva iteración.

\subsection{Sentencia break}
La sentencia \lstinline{break}, por el contrario, hace que el control abandone
definitivamente el lazo:

\begin{lstlisting}
while(expresión) {
    if(ya_no_procesar())
        break;
    procesar();
}
seguir();
\end{lstlisting}

Cuando la función \lstinline{ya_no_procesar()} dé distinto de cero, el control saltará a la
función \lstinline{seguir()}, terminando la ejecución de la estructura repetitiva.

\subsection{Sentencia goto}
Un \textbf{rótulo} es un nombre, seguido del carácter ":", que se asocia a un segmento
de un programa. La sentencia \lstinline{goto} transfiere el control a la instrucción
siguiente a un rótulo. Aunque no promueve la programación estructurada, y se
sabe que su abuso es perjudicial, \lstinline{goto} es útil para resolver algunas
situaciones. Por ejemplo: anidamiento de lazos con salida forzada.

\begin{lstlisting}
for(i=0; i<10; i++) {
    for(j=0; j<50; j++) {
        if(ya_no_procesar(i,j))
            goto final;
        procesar(i,j);
    }
}
final: imprimir(i,j);
\end{lstlisting}

Aquí se podría implementar una estrategia estructurada, en base a \lstinline{break}, pero
el control quedaría retenido en el lazo exterior y se requeriría más lógica
para resolver este problema. Se complicaría la legibilidad del programa
innecesariamente.

Los rótulos a los que puede dirigirse un \lstinline{goto} tienen un espacio de nombres
propio. Es decir, no hay peligro de conflicto entre un rótulo y una variable
del mismo nombre. Además, el ámbito de un rótulo es local a la función (una
sentencia \lstinline{goto} sólo puede acceder a los rótulos dentro del texto de la función
donde aparece).

\subsection{Sentencia return}
Permite devolver un valor a la función llamadora. Implica una transferencia de
control incondicional hasta el punto de llamada de la función que se esté
ejecutando.

\section{Observaciones}
Hay errores de programación típicos, relacionados con estructuras de control en
C, que vale la pena enumerar:
\begin{itemize}
	\item Terminar el encabezado de las estructuras de control con un punto y coma
      extra.
    \item Olvidar la sentencia \lstinline{break} separando casos de un \lstinline{switch}.
    \item Confundir el significado de un lazo \lstinline{do-while} tomando la condición de
      mientras como si fuera una condición de hasta (por analogía con \lstinline{repeat} de
      Pascal).
\end{itemize}
    


\begin{preguntas}
\label{tc-control-preg}
\question ¿Qué resultado final tiene la variable \code{a}?
\begin{lstlisting}
a = 1; 
if(a == 1)
	a = 2;
\end{lstlisting}
\choice 1.
\correctchoice 2.
\choice Ninguna de las anteriores.

\question ¿Qué resultado final tiene la variable \code{a}?
\begin{lstlisting}
a = 1; 
if(3)
	a = 2;
\end{lstlisting}
\choice 1.
\correctchoice 2.
\choice Ninguna de las anteriores.

\question ¿Qué resultado final tiene la variable \code{a}?
\begin{lstlisting}
a = 1;
if(b == 2)
	a = 2;
\end{lstlisting}
\choice 1.
\choice 2.
\correctchoice Depende del valor de b.
\choice Ninguna de las anteriores.

\question ¿Qué resultado final tiene la variable \code{a}?
\begin{lstlisting}
a = 1;
if(b == 2);
	a = 2;
\end{lstlisting}
\choice 1.
\correctchoice 2.
\choice Depende del valor de b.
\choice Ninguna de las anteriores.

\question ¿Qué resultado final tiene la variable \code{a}?
\begin{lstlisting}
a = 1;
if(b = 0)
    a=2;
\end{lstlisting}
\correctchoice 1.
\choice 2.
\choice Depende del valor de b.
\choice Ninguna de las anteriores.

\question ¿Qué resultado final tiene la variable \code{a}?
\begin{lstlisting}
b = 3;
if(b == 1)
	a=2;
else 
	if(b == 2)
	 	a=3; 
	else 
		a=4;
\end{lstlisting}
\choice 2. 
\choice 3.
\correctchoice 4.
\choice No está definido.

\question ¿Qué resultado final tiene la variable \code{a} si inicialmente a, c y d valen 1?
\begin{lstlisting}
switch(c) {
	case 1: a = a+d;  
		    break;
	case 2: a = a-d;
			break;
}
\end{lstlisting}
\choice 1.
\correctchoice 2.
\choice 3.

\question ¿Qué resultado final tiene la variable \code{a} si inicialmente a, c y d valen 1?
\begin{lstlisting}
switch(c) {
	case 1: a = a+d;  
	case 2: a = a-d;
}
\end{lstlisting}
\correctchoice 1.
\choice 2.
\choice 3.

\question ¿Qué resultado final tiene la variable \code{b} si inicialmente b, c y d valen 1?
\begin{lstlisting}
switch(c) {
	case 1: b = b+d;
	case 2: b = b-d;
	default: b = 0;
}
\end{lstlisting}
\correctchoice 0
\choice 1
\choice 2
\choice 3

\question ¿Qué resultado final tiene la variable \code{b} si inicialmente b, c y d valen 1?
\begin{lstlisting}
switch(c) {
	case 1: b = b+d;
			break;
	case 2: b = b-d;
			break;
	default: b = 0;
}
\end{lstlisting}
\choice 0
\choice 1
\correctchoice 2
\choice 3

\question ¿Qué resultado final tiene la variable \code{b} si inicialmente b, c y d valen 3?
\begin{lstlisting}
switch(c) {
	case 1: b = b+d;
			break;
	case 2: b = b-d;
			break;
	default: b = 0;
}
\end{lstlisting}
\correctchoice 0
\choice 1
\choice 2
\choice 3

\question ¿Qué resultado final tiene la variable \code{c}?
\begin{lstlisting}
c = 1;
for(i=0; i<5; i++);
	for(j=0; j<2; j++)    
		c++;
\end{lstlisting}
\choice 1
\choice 2
\correctchoice 3
\choice 6
\choice 13

\question ¿Cuántas X imprimen estas líneas?
\begin{lstlisting}
c = 3; 
do {
	printf("X");
} while(c--);
\end{lstlisting}
\choice 1
\choice 2
\choice 3
\correctchoice 4

\question ¿Cuántas X imprimen estas líneas?
\begin{lstlisting}
c = 3; 
do { 
	printf("X"); 
} while(--c);
\end{lstlisting}
\choice 1
\choice 2
\correctchoice 3
\choice 4

\question ¿Cuántas X imprimen estas líneas?
\begin{lstlisting}
c = 3; 
while(c--) 
	printf("X"); 
\end{lstlisting}
\choice 1
\choice 2
\correctchoice 3
\choice 4

\question ¿Cuántas X imprimen estas líneas?
\begin{lstlisting}
c = 3; 
while(--c) 
	printf("X");
\end{lstlisting}
\choice 1
\correctchoice 2
\choice 3
\choice 4

\question ¿Cuántas X imprimen estas líneas?
\begin{lstlisting}
c = 3; 
while(--c); 
	printf("X");
\end{lstlisting}
\correctchoice 1
\choice 2
\choice 3
\choice 4
\end{preguntas}


\section{Ejercicios}
\label{tc-control-ej}

\begin{enumerate}
\item Reescribir estas sentencias usando while en vez de for:
\begin{enumerate}[label=\alph*.]
	\item 
\begin{lstlisting}
for(i=0; i<=10; i++)
    a = i;
\end{lstlisting}

\item 
\begin{lstlisting}
for( ; j<100; j+=2) {
    a = j;
    b = j * 2;
}
\end{lstlisting}

\item 
\begin{lstlisting}
for( ; ; )
    a++;
\end{lstlisting}
\end{enumerate}

\item Si la función \lstinline{quedanDatos()} devuelve el valor lógico que sugiere su nombre,
¿cuál es la estructura preferible?
\begin{enumerate}[label=\alph*.]
\item 
\begin{lstlisting}
while(quedanDatos()) {
    procesar();
}
\end{lstlisting}

\item 
\begin{lstlisting}
do {
    procesar();
} while(quedanDatos());
\end{lstlisting}
\end{enumerate}
\item ¿Cuál es el error de programación en estos ejemplos?
	\begin{enumerate}[label=\alph*.]
\item 
\begin{lstlisting}
for(i = 0; i < 10; i++);
    a = i - 50L;
\end{lstlisting}

\item 
\begin{lstlisting}
while(i < 100) {
    procesar(i);
    a = a + i;
}
\end{lstlisting}
\end{enumerate}
\item ¿Cuál es el valor de x a la salida de los lazos siguientes?
	\begin{enumerate}[label=\alph*.]
	\item 
\begin{lstlisting}
for(x = 0; x<100; x++);
\end{lstlisting}

	\item 
\begin{lstlisting}
for(x = 32; x<55; x += 3);
\end{lstlisting}

	\item 
\begin{lstlisting}
for(x =  10;x>0; x--);
\end{lstlisting}
\end{enumerate}
\item ¿Cuántas X escriben estas líneas?
\begin{lstlisting}
for (x = 0; x < 10; x++)
    for (y = 5;  y >  0; y--)
        escribir("X");
\end{lstlisting}
\item Escribir sentencias que impriman la tabla de multiplicar para un entero
dado.
\item Imprimir la tabla de los diez primeros números primos (sólo divisibles por
sí mismos y por la unidad).
\item Escribir las sentencias para calcular el factorial de un entero.
\end{enumerate}
%TODO Ejercicios_Adicionales Ejercicios_Avanzados


