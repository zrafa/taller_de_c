\begin{preguntas}
\label{tc-operadores-preg}
\question El operador \code{^} en C significa
\choice exponenciación en base 10.
\choice exponenciación en base $e$.
\correctchoice \textbf{or} exclusivo de bits.
\choice \textbf{or} lógico exclusivo.

\question Luego de ejecutar las sentencias 
\begin{lstlisting}
c = 1; 
a = c++;	
\end{lstlisting}
las variables \code{a} y \code{c} valen respectivamente:
\choice 1 y 1.
\correctchoice 1 y 2.
\choice 2 y 2.
\choice 2 y 1.
\choice ninguna de las anteriores.

\question Luego de ejecutar las sentencias 
\begin{lstlisting}
c = 1; 
a = ++c;
\end{lstlisting}
las variables \code{a} y \code{c} valen respectivamente:
\choice 1 y 1.
\choice 1 y 2.
\correctchoice 2 y 2.
\choice 2 y 1.
\choice ninguna de las anteriores.

\question Luego de ejecutar las sentencias 
\begin{lstlisting}
c = 1; 
a = --c; 
a += c++;
\end{lstlisting}
las variables \code{a} y \code{c} valen respectivamente
\choice 1 y 1.
\choice 1 y 2.
\choice 2 y 2.
\choice 2 y 1.
\correctchoice ninguna de las anteriores.

\question La sentencia \code{a = a \% 2;} puede escribirse también
\choice \code{a\%\%;}
\correctchoice \code{a \%= 2;}
\choice \code{a =\% 2;}
\choice \code{a\%2;}

\question Luego de ejecutar las sentencias 
\begin{lstlisting}
a = 1; 
b = 2; 
if(a == b) 
	b = a;	
\end{lstlisting}
las variables \code{a} y \code{b} valen respectivamente 
\choice 1 y 1.
\correctchoice 1 y 2.
\choice 2 y 2.
\choice ninguno de los anteriores.

\question Luego de ejecutar las sentencias 
\begin{lstlisting}
a = 1; 
b = 2; 
if(a = b) 
	b = a;	
\end{lstlisting}
las variables \code{a} y \code{b} valen respectivamente 
\choice 1 y 1.
\choice 1 y 2.
\correctchoice 2 y 2.
\choice ninguno de los anteriores.

\question Luego de ejecutar las sentencias 
\begin{lstlisting}
a = 1; 
b = 0; 
if(a = b) 
	b = a;
\end{lstlisting}
las variables \code{a} y \code{b} valen respectivamente 
\correctchoice 0 y 0.
\choice 0 y 2.
\choice 2 y 2.
\choice ninguno de los anteriores.

\question ¿Cuál de las reglas \textbf{no es} válida?
\choice Toda expresión cuyo valor aritmético es 0 tiene valor lógico falso.
\choice Toda expresión cuyo valor lógico es falso tiene valor aritmético 0.
\choice Toda expresión cuyo valor aritmético es 1 tiene valor lógico verdadero.
\correctchoice Toda expresión cuyo valor lógico es verdadero tiene valor aritmético 1.

\question Indicar cuál de las expresiones tiene valor lógico falso:
\choice \code{a == a}
\choice \code{2 * a - a}
\correctchoice \code{a = 0}
\choice \code{1 != 0}

\question ¿Cuál es el valor de la expresión \lstinline{c = 20}?
\choice Depende del valor de c.
\correctchoice 20.

\question ¿Cuál es el valor de la expresión \lstinline{c == 20}?
\correctchoice Depende del valor de c.
\choice 20.

\question La operación \code{a &= 0x07} equivale a:
\choice dividir a \code{a} por 7.
\choice dividir a \code{a} por 8.
\choice tomar el resto de dividir a \code{a} por 7.
\correctchoice tomar el resto de dividir \code{a} por 8.
\choice restarle 8 a \code{a}.
\choice restarle 7 a \code{a}.

\question La operación \code{a >>= 2} equivale a:
\choice dividir a \code{a} por dos.
\correctchoice dividir a \code{a} por cuatro.
\choice multiplicar a \code{a} por dos.
\choice multiplicar a \code{a} por cuatro.

\question Dada la declaración \code{unsigned char a=1;} la operación \code{a <<= a} tiene como resultado
\choice 0.
\choice 1.
\correctchoice 2.
\choice 255.
\choice 127.

\question La expresión \code{(a == b) ? c : d} vale
\choice a, si a es igual a b.
\choice b, si c es distinto de d.
\choice c, si c es igual a d.
\correctchoice d, si a es distinto de b.

\question La sentencia \code{printf("\%d\\n", (1 != 2) ? 3 : 4);} imprime:
\choice 1.
\choice 2.
\correctchoice 3.
\choice 4.

\end{preguntas}
