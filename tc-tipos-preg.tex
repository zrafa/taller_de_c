
\begin{preguntas}
\label{sec:tc-tipos-preg}
\question Un objeto de datos es
\choice Un tipo de datos.
\choice Un tipo de datos abstracto.
\choice Una variable.
\correctchoice Un espacio de almacenamiento para contener valores.

\question Un objeto de datos es ocupado
\choice Al terminar la ejecución del programa.
\correctchoice Al calcular cada subexpresión.
\choice Al inicio de cada función.

\question El tipo de las expresiones
\choice Es asignado por el compilador.
\choice Es asignado por el usuario.
\correctchoice Las dos anteriores.
\choice Es asignado por el linkeditor.
\choice No puede ser modificado por el usuario.

\question ¿Cuál de las declaraciones siguientes \textbf{no} es correcta?
\choice \code{char byte;}
\choice \code{unsigned char integer;}
\correctchoice \code{unsigned double a;}
\choice \code{long UNO;}
\choice \code{long int eme;}

\question ¿Cuál de las declaraciones siguientes es incorrecta?
\choice \code{int i,j,k;}
\choice \code{char uvw;}
\correctchoice \code{unsigned a, short b;}
\choice \code{unsigned long int integer;}

\question ¿Cuál de las declaraciones siguientes es correcta?
\correctchoice \code{long size;}
\choice \code{double float a;}
\choice \code{unsigned long integer p;}
\choice \code{LONG alfa;}

\question Una declaración \code{signed} indica que la variable contendrá
\choice un número negativo o cero.
\choice un número positivo o negativo.
\choice un número no negativo.
\correctchoice un número negativo, positivo o cero.

\question El signo default de las variables de tipo \code{int} y de tipo \code{char} es, respectivamente,
\choice \code{signed} y \code{unsigned}.
\correctchoice \code{signed} y dependiente de la implementación.
\choice \code{unsigned} y \code{signed}.
\choice dependiente de la implementación y \code{unsigned}.
\choice dependiente de la implementación en ambos casos.

\question En la declaración siguiente, ¿cuál es el tipo básico interpretado por el compilador? \\
\code{   unsigned short byte;}
\choice char
\correctchoice int
\choice long

\question ¿Cuál es la regla verdadera para los tamaños de los objetos de datos?
\choice Un \code{long} es mayor que un \code{short}.
\choice Un \code{int} es menor que un \code{long}.
\choice Un \code{short} no es menor que un \code{long}.
\correctchoice Un \code{short} no es mayor que un \code{long}.

\question El valor máximo de un \code{unsigned char} suele estar en el orden de
\choice las decenas.
\correctchoice los cientos.
\choice los miles.
\choice las decenas de miles.
\choice los millones.

\question Cuando existe una condición de \textit{overflow}:
\choice El programa aborta con mensaje de error.
\choice El programa es terminado por el subsistema de protección del sistema operativo.
\correctchoice El valor se trunca.
\choice El valor se redondea.
\choice La variable vuelve a cero.

\question Si un \code{signed char} vale 127 y se le suma 1:
\choice queda en 0.
\choice queda en -127.
\choice queda en 128.
\correctchoice queda en -128.
\choice queda en -1.

\question Si un \code{unsigned int} vale 0 y se le resta 1,
\choice queda en 0.
\choice queda en -1.
\correctchoice queda en el valor del máximo entero sin signo.
\choice queda en 65535.
\choice queda en 32768.

\question Si a y b son enteros, para obtener el valor de su cociente con decimales se debe escribir
\choice \code{a \% b;}
\choice \code{a / b;}
\correctchoice \code{(float) a / b;}
\choice \code{float (a) / b;}
\choice \code{(float)(a / b);}

\question Si a es int y b es \code{long}, ¿cuál es el tipo de la expresión \code{a / b}?
\correctchoice \code{long}
\choice \code{int}
\choice \code{float}

\question Si a y b son \code{char}s que tienen el máximo valor posible para los chars, el tipo de la expresión \code{a * b} es:
\choice \code{unsigned char}
\choice \code{unsigned}
\choice \code{int}
\correctchoice \code{char}
\end{preguntas}
