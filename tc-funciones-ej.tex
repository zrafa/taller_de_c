
\section{Ejercicios}
\label{sec:tc-funciones-ej}
\begin{enumerate}
	\item Escribir una función que reciba tres argumentos enteros y devuelva un
entero, su suma.
\item Escribir una función que reciba dos argumentos enteros y devuelva un long,
su producto.
\item Escribir una función que reciba dos argumentos enteros a y b, y utilice a
las dos anteriores para calcular:
\begin{lstlisting}
(a * b + b * 5 + 2) * (a + b + 1)
\end{lstlisting}
\item Escribir un programa que utilice la función anterior para realizar el
cálculo con a=7 y b=3.
\item ¿Qué está mal en estos ejemplos?
\begin{enumerate}[label=\alph*.]
\item 
\begin{lstlisting}
int f1(int x, int y);
{
    int z;
    z = x - y;
    return z;
}
\end{lstlisting}

\item 
\begin{lstlisting}
void f2(int k)
{
    return k + 3;
}
\end{lstlisting}

\item 
\begin{lstlisting}
int f3(long k)
{
    return (k < 0) ? -1 : 1;
}
printf("%d\n",f3(8));
\end{lstlisting}
\end{enumerate}
\item Escribir una función que reciba dos argumentos, uno de tipo int y el otro de
tipo char. La función debe repetir la impresión del char tantas veces como lo
diga el otro argumento. Escribir un programa para probar la función.

\end{enumerate}

%TODO Ejercicios_Adicionales Ejercicios_Avanzados


