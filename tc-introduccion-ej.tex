\section{Ejercicios}
\label{sec:tc-introduccion-ej}
\begin{enumerate}
	\item ¿Qué nombres son adecuados para los archivos fuente C? 
	\item Describa las etapas del ciclo de compilación.
	\item ¿Cuál sería el resultado de: 
		\begin{itemize}
		\item Editar un archivo fuente? 
		\item Ejecutar un archivo fuente? 
		\item Editar un archivo objeto? 
		\item Compilar un archivo objeto? 
		\item Editar una biblioteca?
		\end{itemize}
	\item ¿Qué pasaría si un programa en C \textbf{no} contuviera una función \code{main()}? Haga la prueba modificando \textbf{hola.c}.
	\item Edite el programa \textbf{hola.c} y modifíquelo según las pautas que siguen. Interprete los errores de compilación. Identifique en qué etapa del ciclo de compilación ocurren los errores. Si resulta un programa ejecutable, observe qué hace el programa y por qué. 
		\begin{itemize}
		\item Quite los paréntesis de \code{main()}. 
		\item Quite la llave izquierda de \code{main()}.
		\item Quite las comillas izquierdas.
		\item Quite los caracteres \quotes{\code{\\n}}.
		\item Agregue al final de la cadena los caracteres \quotes{\code{\\n\\n\\n\\n}}.
		\item Agregue al final de la cadena los caracteres \quotes{\code{\\nAdiós, mundo!\\n}}.
		\item Quite las comillas derechas.
		\item Quite el signo punto y coma. 
		\item Quite la llave derecha de \code{main()}.
		\item Agregue un punto y coma en cualquier lugar del texto.
		\item Agregue una coma o un dígito en cualquier lugar del texto. 
		\item Reemplace la palabra \code{main} por \code{program}, manteniendo los paréntesis. 
		\item Elimine la apertura o cierre de los comentarios.
		\end{itemize}
\end{enumerate}
