
\documentclass[spanish,11pt,a4paper]{book}
\usepackage[utf8x]{inputenc}
\usepackage[T1]{fontenc}
\usepackage[spanish]{babel}
\usepackage{amsmath}
\usepackage{amssymb,amsfonts,textcomp}
\usepackage{array}
\usepackage{multirow}
\usepackage{hhline}
\usepackage{float}
\usepackage{xkeyval}
\usepackage[pdftex]{graphicx}
\usepackage[yyyymmdd,hhmmss]{datetime}
\usepackage[usenames,dvipsnames]{xcolor}
\usepackage{appendix}
\usepackage{enumitem}
\usepackage{fancyvrb}
\usepackage{hyperref}

\usepackage{sectsty}
\definecolor{bl}{rgb}{0.0,0.2,0.6} 
\allsectionsfont{\color{bl}\scshape\selectfont}

% Párrafos sin indentación
\usepackage{parskip} % se encarga de espacios entre párrafos
% Otra forma sin espacios entre párrafos
% \setlength\parindent{0pt}
\raggedbottom

% Titulos de secciones
%-----------------------------------------------------------------
\usepackage{titlesec}




\titleformat{\chapter}
 % {\normalfont\LARGE\bfseries} % Formato de la fuente
 {\normalfont\huge\bfseries} % Formato de la fuente
 {\color{blue!60}Capítulo \thechapter} % "Capítulo X" en celestito{Capítulo \thechapter} % Número del capítulo
 % {Capítulo \thechapter} % Número del capítulo
 {1em} % Espacio entre número y título
 {\color{blue!60}} % Color del título (azul clarito)

% RAFA \titleformat{\section} {\normalfont\Large\bfseries\color{bl}}{\thesection}{1em}{}[{\titlerule[0.8pt]}] 

% Fix TOC separation between numbers and section titles
\usepackage[titles]{tocloft}
\setlength{\cftsecnumwidth}{2.8em}% Set length of number width in ToC for \section
\setlength{\cftsubsecnumwidth}{2.8em}% Set length of number width in ToC for \subsection

\usepackage[tocindentauto]{tocstyle}
\usetocstyle{standard}
% Falta modificar color del titulo
% RAFA \addto\captionsspanish{\renewcommand*\contentsname{Contenidos}}

%
% Caracteres nacionales en listados
%----------------------------------------
%\usepackage{listings}
\usepackage{listingsutf8}
\lstset{literate=
  {á}{{\'a}}1 {é}{{\'e}}1 {í}{{\'i}}1 {ó}{{\'o}}1 {ú}{{\'u}}1
  {Á}{{\'A}}1 {É}{{\'E}}1 {Í}{{\'I}}1 {Ó}{{\'O}}1 {Ú}{{\'U}}1
  {à}{{\`a}}1 {è}{{\`e}}1 {ì}{{\`i}}1 {ò}{{\`o}}1 {ù}{{\`u}}1
  {À}{{\`A}}1 {È}{{\'E}}1 {Ì}{{\`I}}1 {Ò}{{\`O}}1 {Ù}{{\`U}}1
  {ä}{{\"a}}1 {ë}{{\"e}}1 {ï}{{\"i}}1 {ö}{{\"o}}1 {ü}{{\"u}}1
  {Ä}{{\"A}}1 {Ë}{{\"E}}1 {Ï}{{\"I}}1 {Ö}{{\"O}}1 {Ü}{{\"U}}1
  {â}{{\^a}}1 {ê}{{\^e}}1 {î}{{\^i}}1 {ô}{{\^o}}1 {û}{{\^u}}1
  {Â}{{\^A}}1 {Ê}{{\^E}}1 {Î}{{\^I}}1 {Ô}{{\^O}}1 {Û}{{\^U}}1
  {œ}{{\oe}}1 {Œ}{{\OE}}1 {æ}{{\ae}}1 {Æ}{{\AE}}1 {ß}{{\ss}}1
  {ç}{{\c c}}1 {Ç}{{\c C}}1 {ø}{{\o}}1 {å}{{\r a}}1 {Å}{{\r A}}1
  {€}{{\EUR}}1 {£}{{\pounds}}1 
  {¡}{{!`}}1 {¿}{{\?`}}1 
  {ñ}{{\~n}}1
}


% Environment ejemplo
%----------------------------------------
\newcounter{ejemplo}[chapter]
\newenvironment{ejemplo}[1][\unskip]{\stepcounter{ejemplo}\footnotesize{\bfseries\color{bl}{\noindent\\Ejemplo~\thechapter.\theejemplo #1}}\medskip\hrule\medskip\noindent}{\par}


\usepackage{multicol}
\definecolor{dkgreen}{rgb}{0,0.6,0}
\definecolor{gray}{rgb}{0.5,0.5,0.5}
\definecolor{mauve}{rgb}{0.58,0,0.82}
\definecolor{lstbackground}{rgb}{0.95,0.95,0.95}
\definecolor{white}{rgb}{100,100,100}

\lstset{
  backgroundcolor=\color{lstbackground},
% language=Bash,
  aboveskip=3mm,
  belowskip=3mm,
  showstringspaces=false,
  columns=flexible,
  basicstyle={\ttfamily\bfseries},
  numbers=none,
%  xleftmargin=0.5cm,
%  xrightmargin=0.5cm,
%  frame=lr,framesep=0.5cm,framerule=0pt,
%  numberstyle=\tiny\color{gray},
%  keywordstyle=\color{blue},
%  commentstyle=\color{dkgreen},
%  stringstyle=\color{mauve},
  breaklines=true,
  breakatwhitespace=true,
  tabsize=4
}


\lstnewenvironment{mylistings}
  {\lstset{language=C,
    backgroundcolor=\color{listingscolor}, % set backgroundcolor
    basicstyle=\footnotesize,% basic font setting
    }%
  }
  {}

% environment para código en celdas de tablas
\lstnewenvironment{codecell}
  {\lstset{basicstyle={\small\ttfamily},% basic font setting
    backgroundcolor=\color{white}}%
  }{}



\usepackage[font=small,skip=1cm]{caption}
\usepackage{verbatim}


% Outline numbering
\setcounter{secnumdepth}{1}
% Reset section numbering between parts
\makeatletter
\@addtoreset{section}{chapter}
\makeatother  
% List styles
\newcommand\liststyleLi{%
\renewcommand\labelitemi{\tiny${\blacksquare}$}
\renewcommand\labelitemii{\tiny${\square}$}
\renewcommand\labelitemiii{\tiny${\circ}$}
\renewcommand\labelitemiv{\tiny${\circ}$}
}
\newcommand\liststyleLii{%
\renewcommand\labelitemi{{\textbullet}}
\renewcommand\labelitemii{${\circ}$}
\renewcommand\labelitemiii{${\blacksquare}$}
\renewcommand\labelitemiv{{\textbullet}}
}
\newcommand\liststyleLiii{%
\renewcommand\labelitemi{{\textbullet}}
\renewcommand\labelitemii{${\circ}$}
\renewcommand\labelitemiii{${\blacksquare}$}
\renewcommand\labelitemiv{{\textbullet}}
}

\liststyleLi

% Page layout (geometry)
\setlength\voffset{-1in}
\setlength\hoffset{-1in}
\setlength\topmargin{2cm}
\setlength\oddsidemargin{2cm}
\setlength\textheight{23.246668cm}
\setlength\textwidth{17.006cm}
\setlength\footskip{26.144882pt}
\setlength\headheight{1.016cm}
\setlength\headsep{0.508cm}

% Footnote rule
\setlength{\skip\footins}{0.119cm}
\renewcommand\footnoterule{\vspace*{-0.018cm}\setlength\leftskip{0pt}\setlength\rightskip{0pt plus 1fil}\noindent\textcolor{black}{\rule{0.25\columnwidth}{0.018cm}}\vspace*{0.101cm}}

% Pages styles
\makeatletter
\newcommand\ps@Standard{
  \renewcommand\@oddhead{{\raggedleft Cabecera \ } {\raggedright \thepage{}}}
  \renewcommand\@evenhead{\@oddhead}
  \renewcommand\@oddfoot{}
  \renewcommand\@evenfoot{\@oddfoot}
  \renewcommand\thepage{\arabic{page}}
}

% \pagestyle{Standard}
\usepackage{fancyhdr}
\pagestyle{fancy}


% F O N T S 
%--------------------------------------
% \usepackage{dejavu}
%\usepackage{librebaskerville}
%\usepackage{sans}
%\usepackage{libertine}
%\usepackage{lmodern}
%\usepackage{opensans}
%\usepackage{helvet}
%\usepackage{times}

% FONT DEFAULT
\usepackage{lmodern}
\renewcommand*\familydefault{\sfdefault} %% Only if the base font of the document is to be sans serif

% Fancy chapter headings

%%%\usepackage[Rejne]{fncychap}
%Must be after setting textwidth
% RAFA \usepackage[Bjornstrup]{fncychap}
%\ChNameVar{\huge\bfseries\textcolor{blue}}
%%%\usepackage[Bjarne]{fncychap}
%%%\usepackage[Glenn]{fncychap}
%%%\usepackage[Conny]{fncychap}
%%%\usepackage[Lenny]{fncychap}

\usepackage{mdframed}

% footnotes configuration
\makeatletter
\renewcommand\thefootnote{\arabic{footnote}}
\makeatother
\usepackage{graphicx}

\usepackage{xkeyval}
\usepackage{pifont}
\newcommand{\revisar}[1]{{\color{red}[#1]}}
%\newcommand{\nota}[1]{{\color{red}[#1]}}
%\newcommand{\revisar}[1]{}

\newcommand{\borrador}{
\revisar{\today, \currenttime  -  Material en preparación, se ruega no imprimir mientras aparezca esta nota}
}

\newcommand{\nota}[1]{}
\newcommand{\nonota}[1]{#1}
\newcommand{\quotes}[1]{``#1''}
\newcommand{\shade}[1]{\textcolor{black!50}{#1}}

% FIGURA
% ancho opcional, por defecto 15cm
% \figura{copyleft}{Símbolo de Copyleft}{copyleft.png}
% \figura[6]{copyleft}{Símbolo de Copyleft}{copyleft.png}
%-----------------------------------------------------------------
\newcommand{\figura}[4][15]{
 \begin{figure}[htb] 
 \centering 
 \includegraphics[width=#1cm]{./img/#4}
 \caption{#3}
 \label{fig:#2} 
 \end{figure} 
}


% TABLAS
% tabla{label}{caption}{columns}{contents}
%-----------------------------------------------------------------
\newcommand{\tabla}[4]{
 \begin{table} 
 \centering 
 \small
 \begin{tabular}{#3}
 #4
 \end{tabular}
 \caption{#2}
 \label{tab:#1} 
 \end{table} 
}


\newcommand{\recuadro}[1]{
\begin{minipage}[c]{0.84\textwidth}
\begin{mdframed}
#1
\end{mdframed}
\end{minipage}
}

\usepackage{float}

% Recuadro importante
%-----------------------------------------------------------------
\newcommand{\importante}[1]{
\bigskip
\noindent
\begin{minipage}[c]{\linewidth}
	\rule{\linewidth}{0.1pt}
	\\{ }\\
     	\begin{minipage}[h]{0.10\linewidth}
        	$\vcenter{\hbox{\includegraphics[width=0.5\linewidth]{./img/emblem-important.png}}}$
     	\end{minipage}
     	\begin{minipage}[h]{0.80\linewidth}\footnotesize
#1
     	\end{minipage}
    \\{ }\\
	\rule{\linewidth}{0.1pt}
\end{minipage}
\bigskip
}


\newcommand{\code}[1]{\lstinline$#1$}




% Multiple choice
%-----------------------------------------------------------------

\newcounter{question}
\newif\ifinchoices
\inchoicesfalse
\newenvironment{questions}{%
  \list{\thequestion.}%
  {%
    \usecounter{question}%
    \def\question{\inchoicesfalse\item}%
    \settowidth{\leftmargin}{10.\hskip\labelsep}%
    \labelwidth\leftmargin\advance\labelwidth-\labelsep
  }%
}
{%
  \endlist
}%


\newcounter{choice}
\renewcommand\thechoice{\alph{choice}}
\newcommand\choicelabel{\thechoice)~}
\def\choice{%
  \ifinchoices
    % Do nothing
  \else
    \startchoices
  \fi
  \refstepcounter{choice}%
  \ifnum\value{choice}>1\relax
  \penalty -50\hskip 1em plus 1em\relax
  \fi
  \choicelabel
  \nobreak\enskip
}% choice
\def\CorrectChoice{%
  \choice
  \addanswer{\thequestion}{\thechoice}%
}
\let\correctchoice\CorrectChoice

\newcommand{\startchoices}{%
  \inchoicestrue
  \setcounter{choice}{0}%
  \par % Uncomment this to have choices always start a new line
  % If we're continuing the paragraph containing the question,
  % then leave a bit of space before the first choice:
  \ifvmode\else\enskip\fi
}%

\newbox\allanswers
\setbox\allanswers=\hbox{}
\newcommand{\addanswer}[2]{%
  \global\setbox\allanswers=\hbox{\unhbox\allanswers #1.~#2\quad}%
}
%\newcommand{\showanswers}{%
%	   \noindent\unhbox\allanswers
%}

\newcommand{\showinvertedanswers}{%
\rotatebox[origin=c]{180} {
	\begin{minipage}[t]{.96\linewidth}
	   \unhbox\allanswers
	\end{minipage}%
}%
}

\newenvironment{preguntas}{%
	\section{Preguntas}
	\begin{questions}
}{%
	\end{questions}
	\subsection{Respuestas}
	\showinvertedanswers
}%
