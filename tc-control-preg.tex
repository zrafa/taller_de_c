
\begin{preguntas}
\label{tc-control-preg}
\question ¿Qué resultado final tiene la variable \code{a}?
\begin{lstlisting}
a = 1; 
if(a == 1)
	a = 2;
\end{lstlisting}
\choice 1.
\correctchoice 2.
\choice Ninguna de las anteriores.

\question ¿Qué resultado final tiene la variable \code{a}?
\begin{lstlisting}
a = 1; 
if(3)
	a = 2;
\end{lstlisting}
\choice 1.
\correctchoice 2.
\choice Ninguna de las anteriores.

\question ¿Qué resultado final tiene la variable \code{a}?
\begin{lstlisting}
a = 1;
if(b == 2)
	a = 2;
\end{lstlisting}
\choice 1.
\choice 2.
\correctchoice Depende del valor de b.
\choice Ninguna de las anteriores.

\question ¿Qué resultado final tiene la variable \code{a}?
\begin{lstlisting}
a = 1;
if(b == 2);
	a = 2;
\end{lstlisting}
\choice 1.
\correctchoice 2.
\choice Depende del valor de b.
\choice Ninguna de las anteriores.

\question ¿Qué resultado final tiene la variable \code{a}?
\begin{lstlisting}
a = 1;
if(b = 0)
    a=2;
\end{lstlisting}
\correctchoice 1.
\choice 2.
\choice Depende del valor de b.
\choice Ninguna de las anteriores.

\question ¿Qué resultado final tiene la variable \code{a}?
\begin{lstlisting}
b = 3;
if(b == 1)
	a=2;
else 
	if(b == 2)
	 	a=3; 
	else 
		a=4;
\end{lstlisting}
\choice 2. 
\choice 3.
\correctchoice 4.
\choice No está definido.

\question ¿Qué resultado final tiene la variable \code{a} si inicialmente a, c y d valen 1?
\begin{lstlisting}
switch(c) {
	case 1: a = a+d;  
		    break;
	case 2: a = a-d;
			break;
}
\end{lstlisting}
\choice 1.
\correctchoice 2.
\choice 3.

\question ¿Qué resultado final tiene la variable \code{a} si inicialmente a, c y d valen 1?
\begin{lstlisting}
switch(c) {
	case 1: a = a+d;  
	case 2: a = a-d;
}
\end{lstlisting}
\correctchoice 1.
\choice 2.
\choice 3.

\question ¿Qué resultado final tiene la variable \code{b} si inicialmente b, c y d valen 1?
\begin{lstlisting}
switch(c) {
	case 1: b = b+d;
	case 2: b = b-d;
	default: b = 0;
}
\end{lstlisting}
\correctchoice 0
\choice 1
\choice 2
\choice 3

\question ¿Qué resultado final tiene la variable \code{b} si inicialmente b, c y d valen 1?
\begin{lstlisting}
switch(c) {
	case 1: b = b+d;
			break;
	case 2: b = b-d;
			break;
	default: b = 0;
}
\end{lstlisting}
\choice 0
\choice 1
\correctchoice 2
\choice 3

\question ¿Qué resultado final tiene la variable \code{b} si inicialmente b, c y d valen 3?
\begin{lstlisting}
switch(c) {
	case 1: b = b+d;
			break;
	case 2: b = b-d;
			break;
	default: b = 0;
}
\end{lstlisting}
\correctchoice 0
\choice 1
\choice 2
\choice 3

\question ¿Qué resultado final tiene la variable \code{c}?
\begin{lstlisting}
c = 1;
for(i=0; i<5; i++);
	for(j=0; j<2; j++)    
		c++;
\end{lstlisting}
\choice 1
\choice 2
\correctchoice 3
\choice 6
\choice 13

\question ¿Cuántas X imprimen estas líneas?
\begin{lstlisting}
c = 3; 
do {
	printf("X");
} while(c--);
\end{lstlisting}
\choice 1
\choice 2
\choice 3
\correctchoice 4

\question ¿Cuántas X imprimen estas líneas?
\begin{lstlisting}
c = 3; 
do { 
	printf("X"); 
} while(--c);
\end{lstlisting}
\choice 1
\choice 2
\correctchoice 3
\choice 4

\question ¿Cuántas X imprimen estas líneas?
\begin{lstlisting}
c = 3; 
while(c--) 
	printf("X"); 
\end{lstlisting}
\choice 1
\choice 2
\correctchoice 3
\choice 4

\question ¿Cuántas X imprimen estas líneas?
\begin{lstlisting}
c = 3; 
while(--c) 
	printf("X");
\end{lstlisting}
\choice 1
\correctchoice 2
\choice 3
\choice 4

\question ¿Cuántas X imprimen estas líneas?
\begin{lstlisting}
c = 3; 
while(--c); 
	printf("X");
\end{lstlisting}
\correctchoice 1
\choice 2
\choice 3
\choice 4
\end{preguntas}
