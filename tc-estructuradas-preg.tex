

\begin{preguntas}
\label{sec:tc-estructuradas-preg}
\question ¿Cuántos elementos tiene el arreglo \code{long trece[12];}?
\choice 11
\correctchoice 12
\choice 13

\question ¿Cuál es la declaración correcta para un arreglo de nueve caracteres?
\choice \code{int chars[9];}
\choice \code{alfa\=char[9];}
\correctchoice \code{char[9];}
\choice \code{char[9] alfa;}
\correctchoice \code{char alfa[9];}

\question ¿Cuántos elementos tiene el arreglo \code{int trece[12] \= \{1, 3, 5\};}? 
\choice 3
\choice 11
\correctchoice 12
\choice 13

\question ¿Cuántos elementos tiene el arreglo \code{int trece[] \= \{1, 3, 5\};}?
\correctchoice 3
\choice 11
\choice 12
\choice 13

\question Con la declaración del arreglo que sigue, ¿cuál de las sentencias es incorrecta? \\
\code{long trece[12] \= \{1, 5, 20L, 35\};}
\choice \code{trece[1]++;}
\correctchoice \code{trece[12]--;}
\choice \code{trece[1] \= trece[0];}
\correctchoice \code{trece[0];}
\choice \code{trece[11] \= 20L;}
\correctchoice \code{20L;}

\question ¿Cuál de estos segmentos de programa es incorrecto?
\choice \code{int alfa[3]; alfa[2++] \= 8;}
\correctchoice \code{int alfa[3]; alfa \= \{1, 2, 8\};}
\choice \code{int alfa[3]; c \= alfa[0]++;}
\choice \code{int alfa[3]; alfa[1] \= alfa[2];}

\end{preguntas}
