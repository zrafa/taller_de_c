\chapter{Operadores}
\label{chap:tc-operadores}

\section{Operadores aritméticos}
El C tiene un conjunto de operadores muy rico, incluyendo algunos operadores que es difícil encontrar en otros lenguajes de programación. Comenzamos por los operadores aritméticos.
\begin{itemize}
	\item $+$, $-$, $*$, $/$ (operaciones aritméticas usuales)
	\item $++$, $--$ (incremento y decremento)
	\item $\%$ (operador módulo)
	\item No existe operador de \textbf{exponenciación} en C. En cambio, está implementada una función \lstinline{pow()} en la Biblioteca Standard.
	\item No existe operador de \textbf{división entera}, opuesto a la división entre números reales, como en Pascal. Sin embargo, la división entre enteros da un entero: el resultado se trunca debido a las reglas de evaluación de expresiones.
\end{itemize}

\begin{ejemplo}
Aquí \lstinline{a} y \lstinline{b} reciben respectivamente el cociente y el resto de dividir un VALOR por 256. Imprimiéndolos podemos ver cómo una CPU, en la cual un \lstinline{unsigned short int} tuviera un tamaño de 16 bits, almacenaría ese VALOR sobre dos bytes.
\begin{lstlisting}
unsigned char a,b;
a=VALOR / 256;
b=VALOR % 256;  	
\end{lstlisting}
\end{ejemplo}  

\begin{ejemplo}
La siguiente división de \lstinline{j} por \lstinline{k} es entera. La variable \lstinline{c} valdrá 1.
\begin{lstlisting}
float c;
int j,k;
j=3; 
k=2;
c=j/k; 
\end{lstlisting}
\end{ejemplo}



\begin{comment}
\begin{ejemplo}
Los operadores de incremento y decremento ($++$ y $--$) equivalen a las sentencias del tipo \lstinline{a = a+1} o \lstinline{b = b-1}. Suman o restan una unidad a su argumento, que debe ser un tipo entero. Se comportan diferentemente según que se apliquen en forma prefija o sufija.


% \centering
% \begin{tabular}{l|l|l}
% \hline
% Sentencias & a & b \\
% \hline
% \begin{codecell}
% a=5; b=a++;
% \end{codecell}
% &
% 6
% &
% 5\\
% \hline
% \begin{codecell}
% a=5; b=++a;
% \end{codecell}
% &
% 6
% &
% 6\\
% \hline
% \begin{codecell}
% a=3; b=a--;
% \end{codecell}
% &
% 2
% &
% 3\\	
% \hline
% \begin{codecell}
% a=3; b=--a;
% \end{codecell}
% &
% 2
% &
% 2\\
% \end{tabular}

\begin{itemize}
	\item Aplicados como prefijos, el valor devuelto por la expresión es el valor incrementado o decrementado.
	\item Aplicados como sufijos, el incremento o decremento se realiza como efecto lateral, pero el valor devuelto por la expresión es el anterior al incremento o decremento.
\end{itemize}
\end{ejemplo}
\end{comment}

\subsection{Abreviaturas}
Existe una forma de abreviar la notación en las expresiones del tipo \lstinline{a = a*b} y \lstinline{a = a+b}. Podemos escribir, respectivamente:
\begin{lstlisting}
a *= b;
a += b; 
\end{lstlisting}

Esto se aplica a todos los operadores aritméticos y de bits.

\section{Operadores de relación}

\begin{itemize}
\item \lstinline{==} (igualdad)
\item \lstinline{<, >, <=, >=}
\item \lstinline{!=} (distinto de)
\end{itemize}

\importante{
\begin{itemize}
\item Es un error muy común sustituir el operador de relación \lstinline{==} por el de asignación \lstinline{=}. 
\item Este error lógico no provocará error de compilación ni de ejecución, sino que será interpretado como una asignación, que en C puede ser hecha en el mismo contexto que una comparación. 
\item Una instrucción de asignación puede funcionar como una expresión en un contexto, y se evalúa al valor asignado.
\end{itemize}
}

La sentencia \lstinline{k=1;}, claramente, asigna 1 a la variable \code{k}; pero además, si se encuentra en un contexto donde sea evaluada, la expresión \lstinline{k=1} vale 1. Como consecuencia, su valor lógico es \lstinline{TRUE}, independientemente del valor que tuviera k. Por lo tanto, la sentencia \lstinline{if(k=1) k=2;} \textbf{siempre} asignará el valor 2 a \code{k}.
Por el contrario, la expresión \lstinline{k=0} es siempre \lstinline{FALSE}.

La sentencia:
\begin{lstlisting}
if(k = 1)
     ...
\end{lstlisting}     
es sintácticamente válida, pero en lugar de \textbf{comparar} \lstinline{k} con 1, \textbf{asigna} el valor 1 a la variable \lstinline{k}. 

% La expresión \lstinline{a=38} es legal en C como expresión lógica. Su valor de verdad es \textbf{siempre} \lstinline{TRUE}, independientemente del valor que tuviera \lstinline{a} anteriormente. Si el programador desea obtener el valor lógico de la comparación de \lstinline{a} con 38, debe utilizar la expresión \lstinline{if(a == 38)}. Por el contrario, la expresión \lstinline{a=0} es siempre \lstinline{FALSE}.
Si el programador desea obtener el valor lógico de la comparación de \lstinline{k} con 1, debe utilizar el operador de relación \lstinline{==}, y la expresión correcta es \lstinline{if(k == 1)}. 



\section{Operadores lógicos}
\begin{lstlisting}
!, &&, || (not, and, or)	
\end{lstlisting}

\begin{ejemplo}
La siguiente es una expresión lógica (\quotes{\lstinline{a} igual a \lstinline{b} o no \lstinline{c} menor que 2}).
\begin{lstlisting}
a==b || !(c < 2)    
\end{lstlisting}
\end{ejemplo}

Los valores lógicos se equiparan a los aritméticos. 
\begin{itemize}
	\item Una expresión formada con un operador \lstinline{||}, \lstinline{&&} o \lstinline{!} arroja un valor aritmético $1$ si es $V$, y un valor $0$ si es $F$. 
	\item Toda expresión cuyo valor aritmético sea diferente de $0$ es \quotes{verdadera}. 
	\item Toda expresión que dé $0$ es \quotes{falsa}.
\end{itemize}

\begin{ejemplo}
\noindent
\begin{itemize}
\item \lstinline{a - b}, que es una expresión aritmética, es también una expresión lógica. Será $F$ cuando a sea igual a b.
\item \lstinline{a} como expresión lógica será $V$ sólo cuando a sea distinto de 0 .

\item \lstinline{a = 8} es una expresión de asignación. Su valor aritmético es el valor asignado, por lo cual, como expresión lógica, es $V$, ya que 8 es distinto de 0.

\end{itemize}
\end{ejemplo}

\subsection{Constantes lógicas}
Generalmente los compiladores definen dos símbolos o macros, \lstinline{FALSE} y \lstinline{TRUE}, en el header \lstinline{stdlib.h}, y a veces también en otros. Su definición suele ser la siguiente:

\begin{lstlisting}
#define FALSE  0
#define TRUE   !FALSE   
\end{lstlisting}
    
Es conveniente utilizar estos símbolos para agregar expresividad a los programas.

\section{Operadores de bits}
\begin{itemize}
	\item \lstinline{~} (negación de bits)
	\item \lstinline{&, |} (\textbf{and}, \textbf{or} de bits)
	\item \lstinline{^} (\textbf{or} de bits exclusivo)
	\item \lstinline{>>, <<} (desplazamiento de bits a derecha y a izquierda)

El desplazamiento de bits es \textbf{con pérdida}, en el sentido de que:
\begin{itemize}
	\item Un desplazamiento a la izquierda en un bit equivale a una multiplicación por 2; un desplazamiento a la derecha en un bit equivale a una división por 2. 
	\item Si el bit menos significativo es 1 (si el número es impar), al desplazar a la derecha ese bit se pierde (la división no es exacta).
	\item Si el bit más significativo es 1 (si el número es igual o mayor que la mitad de su rango posible), al desplazar a la izquierda ese bit se pierde (la multiplicación da overflow).
\end{itemize}
\end{itemize}

\begin{ejemplo}
El operador de \textbf{negación} de bits es unario y provoca el complemento a uno de su operando. Los operadores \textbf{and}, \textbf{or} y \textbf{or exclusivo} de bits son binarios.
\begin{lstlisting}
unsigned char a;
a = ~255;             /* a <-- 0    */
a = 0xf0 ^ 255;       /* a <-- 0x0f */
a = 0xf0 & 0x0f;      /* a <-- 0x00 */
a = 0xf0 | 0x0f;      /* a <-- 0xff */                
\end{lstlisting}
\end{ejemplo}

\begin{ejemplo}
Los operadores \lstinline{>>} y \lstinline{<<} desplazan los bits de un objeto de tipo \lstinline{char}, \lstinline{int} o \lstinline{long}, una cantidad dada de posiciones. 
\begin{lstlisting}
unsigned char a,b,c;
a = 1 << 4;
b = 2 >> 1;
c <<= 2;    
\end{lstlisting}
\begin{itemize}
	\item En el primer caso, el 1 se desplaza 4 bits a la izquierda; \lstinline{a} vale finalmente 16. 
	\item En el segundo caso, el bit 1 de la constante 2, a uno, se desplaza un lugar a la derecha; \lstinline{b} vale 1. 
	\item En el tercero, \lstinline{c} se desplaza dos bits a la izquierda. 
\end{itemize}
\end{ejemplo}


\section{Operadores especiales}

Distinguimos como especiales los operadores de \textbf{asignación} e \textbf{inicialización}, que son operaciones diferentes aunque desafortunadamente son representadas por el mismo signo, y el operador \textbf{ternario}.

\begin{itemize}
	\item \lstinline{=} (operador de asignación)
	\item \lstinline{=} (operador de inicialización)
	\item ?: (operador ternario)
\end{itemize}


\subsection{Inicialización}
Una inicialización es la operación que permite asignar un valor inicial a una variable en el momento de su definición: 
\begin{lstlisting}
int a=1;
\end{lstlisting}


\subsection{Asignación}
Una asignación es la operación que permite asignar un valor a una variable que ha sido definida anteriormente: 
\begin{lstlisting}
int a;
a=1;
\end{lstlisting}

\subsection{Operador ternario}
El operador ternario comprueba el valor lógico de una expresión, y, según este valor, se evalúa a una u otra de las restantes expresiones. Supongamos tener la expresión siguiente.

\begin{lstlisting}
a = (expresión_1) ? expresión_2 : expresión_3;    
\end{lstlisting}

Entonces, si \lstinline{expresión_1} es $V$, el ternario se evaluará a \lstinline{expresión_2}, y ese valor será asignado a la variable \lstinline{a}. Si \lstinline{expresión_1} es $F$, \lstinline{a} quedará con el valor \lstinline{expresión_3}.

\begin{ejemplo}
La expresión
\begin{lstlisting}
c = b + (a < 0) ? -a : a;   
\end{lstlisting}      
asigna a \lstinline{c} el valor de \lstinline{b} más el valor absoluto de \lstinline{a}.
\end{ejemplo}

\section{Precedencia y orden de evaluación}

En una expresión compleja, formada por varias subexpresiones conectadas por operadores, es peligroso hacer depender el resultado del orden de evaluación de las subexpresiones. Si los operadores en una subexpresión tienen la misma precedencia, la evaluación se hace de izquierda a derecha, pero en caso contrario el orden de evaluación no queda definido. 
%TODO ACLARAR EL EJEMPLO
Por ejemplo, en la expresión \lstinline{w*x/++y + z/y} se puede contar con que primeramente se ejecutará \lstinline{w*x} y sólo entonces \lstinline{w*x/++y}, porque los operadores de multiplicación y división son de la misma precedencia. Sin embargo, no se puede asegurar que \lstinline{w*x/++y} sea evaluado antes o después de \lstinline{z/y}, lo que hace que el resultado de esta expresión sea \textbf{indefinido} en C, ya que no sabemos cuál será el valor de \lstinline{y} que intervendrá en el cómputo de \lstinline{z/y}. 

La solución es \textbf{secuencializar} la ejecución, dividiendo las expresiones en sentencias:
\begin{lstlisting}
a = w * x / ++y;
b = a + z / y;     	
\end{lstlisting}
 
\section{Resumen}

El Cuadro \ref{tab:preced} ilustra las relaciones de precedencia entre los diferentes operadores. La tabla está ordenada con los operadores de mayor precedencia a la cabeza. Los que se encuentran en el mismo renglón tienen la misma precedencia.

\begin{comment}
\tabla{preced}{Relaciones de precedencia entre operadores}{c|l|l}
{
\hline
Precedencia & Operador &\\
\hline
1 &\lstinline{()} & Llamada a función\\
  & \lstinline{[]} & Indexación de arreglo\\ 
 &\lstinline{-> .} & Acceso a miembros de estructuras\\
 \hline
2 &\lstinline{! ~} & Negación y complemento a uno\\
 &\lstinline{++ --}& Incremento y decremento\\
 &\lstinline{*} & Indirección \\
  &\lstinline{&} & Dirección de\\
 &\lstinline{()} & Conversión forzada de tipo (\textit{cast})\\
 &\lstinline{sizeof} & Tamaño de variables o tipos\\
 &\lstinline{+} unario &\\
  &\lstinline{-} unario &\\
 \hline
3 &\lstinline{* / \%}& Multiplicación, división y módulo\\
 \hline
4 &\lstinline{+ -} & Suma y resta\\
 \hline
5 &\lstinline{\<\< \>\>} & Desplazamiento de bits\\
 \hline
6 &\lstinline{\< \<= \>= \>} & Operadores de relación\\
 \hline
7 &\lstinline{== !=} & Operadores lógicos de igualdad \\
 \hline
8 &\lstinline{&} &AND de bits\\
 \hline
9 &\lstinline{^} &XOR de bits\\
 \hline
10 &\lstinline{|} &OR de bits\\
 \hline
11&\lstinline{&&}& AND lógico\\
 \hline
12&\lstinline{||}& OR lógico\\
 \hline
13&\lstinline{?:}& Operador ternario\\
&\lstinline{=} &Asignación\\
 \hline
14 &\lstinline{+= -= *= /= \%=}& Aritméticos y lógicos abreviados\\
&\lstinline{&= ^= |=} &\\
&\lstinline{\<\<= \>\>=} & \\
 \hline
15&\lstinline{,} & Operador coma\\
}
\end{comment}


\begin{preguntas}
\label{tc-operadores-preg}
\question El operador \code{^} en C significa
\choice exponenciación en base 10.
\choice exponenciación en base $e$.
\correctchoice \textbf{or} exclusivo de bits.
\choice \textbf{or} lógico exclusivo.

\question Luego de ejecutar las sentencias 
\begin{lstlisting}
c = 1; 
a = c++;	
\end{lstlisting}
las variables \code{a} y \code{c} valen respectivamente:
\choice 1 y 1.
\correctchoice 1 y 2.
\choice 2 y 2.
\choice 2 y 1.
\choice ninguna de las anteriores.

\question Luego de ejecutar las sentencias 
\begin{lstlisting}
c = 1; 
a = ++c;
\end{lstlisting}
las variables \code{a} y \code{c} valen respectivamente:
\choice 1 y 1.
\choice 1 y 2.
\correctchoice 2 y 2.
\choice 2 y 1.
\choice ninguna de las anteriores.

\question Luego de ejecutar las sentencias 
\begin{lstlisting}
c = 1; 
a = --c; 
a += c++;
\end{lstlisting}
las variables \code{a} y \code{c} valen respectivamente
\choice 1 y 1.
\choice 1 y 2.
\choice 2 y 2.
\choice 2 y 1.
\correctchoice ninguna de las anteriores.

\question La sentencia \code{a = a \% 2;} puede escribirse también
\choice \code{a\%\%;}
\correctchoice \code{a \%= 2;}
\choice \code{a =\% 2;}
\choice \code{a\%2;}

\question Luego de ejecutar las sentencias 
\begin{lstlisting}
a = 1; 
b = 2; 
if(a == b) 
	b = a;	
\end{lstlisting}
las variables \code{a} y \code{b} valen respectivamente 
\choice 1 y 1.
\correctchoice 1 y 2.
\choice 2 y 2.
\choice ninguno de los anteriores.

\question Luego de ejecutar las sentencias 
\begin{lstlisting}
a = 1; 
b = 2; 
if(a = b) 
	b = a;	
\end{lstlisting}
las variables \code{a} y \code{b} valen respectivamente 
\choice 1 y 1.
\choice 1 y 2.
\correctchoice 2 y 2.
\choice ninguno de los anteriores.

\question Luego de ejecutar las sentencias 
\begin{lstlisting}
a = 1; 
b = 0; 
if(a = b) 
	b = a;
\end{lstlisting}
las variables \code{a} y \code{b} valen respectivamente 
\correctchoice 0 y 0.
\choice 0 y 2.
\choice 2 y 2.
\choice ninguno de los anteriores.

\question ¿Cuál de las reglas \textbf{no es} válida?
\choice Toda expresión cuyo valor aritmético es 0 tiene valor lógico falso.
\choice Toda expresión cuyo valor lógico es falso tiene valor aritmético 0.
\choice Toda expresión cuyo valor aritmético es 1 tiene valor lógico verdadero.
\correctchoice Toda expresión cuyo valor lógico es verdadero tiene valor aritmético 1.

\question Indicar cuál de las expresiones tiene valor lógico falso:
\choice \code{a == a}
\choice \code{2 * a - a}
\correctchoice \code{a = 0}
\choice \code{1 != 0}

\question ¿Cuál es el valor de la expresión \lstinline{c = 20}?
\choice Depende del valor de c.
\correctchoice 20.

\question ¿Cuál es el valor de la expresión \lstinline{c == 20}?
\correctchoice Depende del valor de c.
\choice 20.

\question La operación \code{a &= 0x07} equivale a:
\choice dividir a \code{a} por 7.
\choice dividir a \code{a} por 8.
\choice tomar el resto de dividir a \code{a} por 7.
\correctchoice tomar el resto de dividir \code{a} por 8.
\choice restarle 8 a \code{a}.
\choice restarle 7 a \code{a}.

\question La operación \code{a >>= 2} equivale a:
\choice dividir a \code{a} por dos.
\correctchoice dividir a \code{a} por cuatro.
\choice multiplicar a \code{a} por dos.
\choice multiplicar a \code{a} por cuatro.

\question Dada la declaración \code{unsigned char a=1;} la operación \code{a <<= a} tiene como resultado
\choice 0.
\choice 1.
\correctchoice 2.
\choice 255.
\choice 127.

\question La expresión \code{(a == b) ? c : d} vale
\choice a, si a es igual a b.
\choice b, si c es distinto de d.
\choice c, si c es igual a d.
\correctchoice d, si a es distinto de b.

\question La sentencia \code{printf("\%d\\n", (1 != 2) ? 3 : 4);} imprime:
\choice 1.
\choice 2.
\correctchoice 3.
\choice 4.

\end{preguntas}

\section{Ejercicios}
\label{tc-operadores-ej}
\begin{enumerate}
\item ¿Qué valor lógico tienen las expresiones siguientes?
\begin{enumerate}[label=\alph*.]
	\item \lstinline{TRUE && FALSE}
	\item \lstinline{TRUE || FALSE}
	\item \lstinline{0 && 1}
	\item \lstinline{0 || 1}
	\item \lstinline{(c > 2) ? (b < 5) : (2 != a)}
	\item \lstinline{(b == c) ? 2 : FALSE;}
	\item \lstinline{c == a;}
	\item \lstinline{C = A;}
	\item \lstinline{0 || TRUE}
	\item \lstinline{TRUE || 2-(1+1)}
	\item \lstinline{TRUE &&  !FALSE}
	\item \lstinline{!(TRUE && !FALSE)}
	\item \lstinline{x == y  >  2}
\end{enumerate}
\item Escriba una macro \lstinline{IDEM(x,y)} que devuelva el valor lógico \lstinline{TRUE} si \lstinline{x} e \lstinline{y} son iguales, y \lstinline{FALSE} en caso contrario. Escriba \lstinline{NIDEM(x,y)} que devuelva \lstinline{TRUE} si las expresiones \lstinline{x} e \lstinline{y} son diferentes y \lstinline{FALSE} si son iguales.

\item Escriba una macro \lstinline{PAR(x)} que diga si un entero es par. Muestre una versión usando el operador \lstinline{%}, una usando el operador \lstinline{>>}, una usando el operador \lstinline{&} y una usando el operador \lstinline{|}.

\item  Escriba macros \lstinline{MIN(x,y)} y \lstinline{MAX(x,y)} que devuelvan el menor y el mayor elemento entre \lstinline{x} e \lstinline{y}. Usando las anteriores, escriba macros \lstinline{MIN3(x,y,z)} y \lstinline{MAX3(x,y,z)} que devuelvan el menor y el mayor elemento entre tres expresiones.

\item ¿Cuál es el significado aritmético de la expresión \lstinline{1<<x} para diferentes valores de \lstinline{x} = 0, 1, 2... ?

\item Utilice el resultado anterior para escribir una macro \lstinline{DOSALA(x)} que calcule $2$ elevado a la $x$-ésima potencia.

\item ¿A qué otra expresión es igual \lstinline{a<b || a<c && c<d}?

\begin{enumerate}[label=\alph*.]
	\item \lstinline{a<b || (a<c && c<d)}
	\item \lstinline{(a<b || a<c) && c<d}
\end{enumerate}

\item Reescribir utilizando abreviaturas:
\begin{enumerate}[label=\alph*.]
	\item \lstinline{a = a + 1;}
	\item \lstinline{b = b * 2;}
	\item \lstinline{b = b - 1;}
	\item \lstinline{c = c - 2;}
	\item \lstinline{d = d % 2;}
	\item \lstinline{e = e & 0x0F;}
	\item \lstinline{a = a + 1;}
	\item \lstinline{b = b + a;}
	\item \lstinline{a = a - 1;}
	\item \lstinline{c = c * a;}
\end{enumerate}

\item ¿Qué escribirá este programa?
\begin{lstlisting}
#include <stdio.h>

int main()
{
	int a = 1;
	int b;

	b = a || 12;

	printf("%d\n",b);
}

\end{lstlisting}
\end{enumerate}

%TODO Ejercicios Adicionales 
%TODO Ejercicios Avanzados


