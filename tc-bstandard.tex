



\chapter{La Biblioteca Standard}
\label{tc-bstandard}



La Biblioteca Standard no forma parte del lenguaje C, estrictamente hablando, pero todos
los compiladores contienen una implementación de ella, a veces con agregados o pequeñas
variantes. Desde la oficialización del ANSI C, los contenidos de la Biblioteca
Standard se han estabilizado y se puede contar con el mismo conjunto de
funciones en todas las plataformas.

Las funciones de la Biblioteca Standard están agrupadas en varias categorías, cada una representada en un header, según el Cuadro \ref{tab:catbs}. Una
implementación de C puede, sin embargo, aportar muchísimos otros headers, más específicos. 

Para poder utilizar cualquiera de las funciones de cada categoría, es necesario
incluir en el fuente el header asociado con la categoría. Esto no
implica incluir los textos de las funciones en la unidad de traducción, sino
simplemente incorporar los prototipos de las funciones de la biblioteca. Es
decir, incluir un header de Biblioteca Standard no \textbf{define} las funciones que se
van a usar, sino que las \textbf{declara}. La resolución de las referencias a las
funciones o variables de Biblioteca Standard quedan pendientes hasta la
linkedición.

Ya hemos visitado la mayoría de las funciones de la categoría de funciones de E/S. Nos
quedan por explorar algunas otras de importancia. Con
esta información no pretendemos reemplazar al manual del compilador, sino
orientar a los primeros pasos en el uso de la Biblioteca Standard.

\tabla{catbs}{Categorías de funciones de la Biblioteca Standard.}{l|l}{
\hline
Categoría & Header \\
\hline
Macros de diagnóstico de errores	&	<assert.h> \\
Macros de clasificación de caracteres	&	<ctype.h> \\
Variables y funciones relacionadas con condiciones de error & <errno.h>\\
Características de la representación en punto flotante	& <float.h>\\
Rangos de tipos de datos, dependientes de la plataforma & <limits.h>\\
Definiciones relacionadas con el idioma y lugar de uso & <locale.h>\\
Funciones matemáticas & <math.h>\\
Saltos no locales & <setjmp.h>\\
Manejo de señales & <signal.h>\\
Listas de argumentos variables & <stdarg.h>\\
Definiciones de algunos tipos y constantes comunes & <stddef.h>\\
Entrada/salida & <stdio.h>\\
Varias funciones útiles & <stdlib.h>\\
Operaciones sobre cadenas & <string.h>\\
Funciones de fecha y hora & <time.h>\\
}


\section{Cadenas o strings}
No existiendo un tipo de datos string en C, se lo implementa como un arreglo de
caracteres, dado por su dirección inicial, y terminado en el carácter especial
\code{'\\0'}. Todas las funciones de strings de Biblioteca Standard, declaradas en el header \code{string.h}, hacen uso de este protocolo de fin
de string. Muchas de ellas han sido implementada en las prácticas de capítulos
anteriores.

En el Cuadro \ref{tab:strings} aparecen las más utilizadas. Es importante consultar también el manual de las funciones \code{strspn()},
\code{strcspn()}, \code{strpbrk()}, \code{strstr()}, \code{strtok()}.

\tabla{strings}{Funciones de Biblioteca Standard para strings.}{l|l|l}{
\hline
Prototipo & Efecto & Ejemplo \\
\hline
\code{char *strcpy(s,ct);} 	& Copia la cadena ct sobre s,  		 & \code{char alfa[10];} 			\\
  							& incluyendo el 0 final. Devuelve s. & \code{strcpy(alfa,"cadena");} 	\\
\hline
\code{char *strncpy(s,ct);} 	& Copia ct sobre s hasta $n$ caracteres. & \code{char alfa[10];}		\\
							& Rellena con 0 hasta el final, si ct 	 & \code{strncpy(alfa,"cadena",4);} \\
							& mide menos de $n$ caracteres. 		 & \code{strncpy(alfa,otro,sizeof(alfa));} \\
\hline
\code{char *strcat(s,ct);} 	& Concatena la cadena ct al final de s.  & \code{char alfa[10]="abc";} \\
							& Debe garantizarse espacio para 		 & \code{strcat(alfa,"def");}\\
							& completar la operación. 				 & \\
\hline
\code{char *strncat(s,ct,n);}& Idem anterior, hasta $n$ caracteres. & \\
\hline
\code{int strcmp(cs,ct);} 	& Compara las cadenas. Devuelve  	& \code{if(strcmp(alfa,"abcdef")==0)} \\
							& un valor menor que 0 si $cs<ct$,  &  \code{\ \ printf("Iguales\\n");} \\
							& 0 si $cs==ct$, y mayor que 0 &  \\
							&  si $cs>ct$. &  \\
\hline
\code{int strncmp(cs,ct,n);} & Idem anterior, hasta $n$ caracteres & \\
\hline
\code{char *strchr(cs,c);} & Devuelve un apuntador a la primera  & \code{char *p;} \\
							& ocurrencia del carácter c en la  & \code{char r[] = "casualidad";} \\
							& cadena cs, o bien NULL &   					\code{p = strchr(r,'s');} \\
							&	si no se lo encuentra.					&						\code{strncpy(p,"us",2);} \\
\hline
\code{char *strrchr(cs,c);} & Idem anterior pero devuelve un & \\
							& apuntador a la última ocurrencia. & \\
\hline
\code{size_t strlen(cs);}   & Devuelve la longitud de la cadena,& \code{for(i=0; i<strlen(s); i++)} \\
							& sin contar el \code{'\0'} final. &  \code{\ \ printf("\%c\\n",s[i]);} \\
}



\section{Listas de argumentos variables}
En C es posible definir funciones con una lista de argumentos de una longitud variable, es decir, 
funciones que reciban una cantidad arbitraria de parámetros reales. La función de Biblioteca 
Standard \code{printf()} es un claro ejemplo de esta clase de funciones, y el usuario puede definir las propias.
Para esto se prepara un encabezado de la función con los parámetros
reales fijos que se desee y se indican los restantes, variables, mediante
puntos suspensivos. Se recuperan los demás con macros especiales definidas en
este header.

Lamentablemente las macros declaradas en el header \code{stdarg.h} no permiten la creación de funciones sin
argumentos fijos. Existe otro paquete de argumentos variables, definido en
\textbf{varargs.h}, que sí lo permite; pero que no está comprendido en el estándar ANSI
C y que no es compatible con \code{stdarg.h}.

\begin{ejemplo}
\begin{lstlisting}
#include <stdarg.h>
int sumar(int cuantos, ...)
{
	va_list ap;
	int suma=0;

	va_start(ap, cuantos);
	for(i=0; i<cuantos; i++)
		suma += va_arg(ap, int);
	va_end(ap);
	return suma;
}
\end{lstlisting}
Que se utilizaría como:
\begin{lstlisting}
#include <stdio.h>

int main()
{
	printf("Resultado 1: %d\n", sumar(3, 4, 5, 6));
	printf("Resultado 2: %d\n", sumar(2, 100, 2336));
}
\end{lstlisting}
\end{ejemplo}

\section{Tratamiento de errores}
Esta zona de la Biblioteca, declarada en los headers \code{errno.h} y \code{assert.h}, provee indispensables herramientas de \textit{debugging}. La
variable externa \code{errno} es un entero que toma un valor de acuerdo a condiciones
de error provocadas por cualquiera de las funciones de la Biblioteca Standard, y de acuerdo a
una catalogación de errores que depende de la función. Si una función ANSI C
devuelve un valor indicador de error (como NULL donde debería devolver un
puntero, o negativo donde debería devolver un positivo), la variable \code{errno}
contendrá más explicación sobre el motivo del error. Se consulta con las
funciones \code{strerror()} o \code{perror()}. La función \code{perror()} admite una cadena
arbitraria para indicar, por ejemplo, el lugar del programa donde se produce el
error. Imprimirá esta cadena más la descripción del problema.

\begin{ejemplo}
\begin{lstlisting}
#include <errno.h>
    if(open("noexiste", O_RDONLY) < 0)
        perror("Error en apertura");	
\end{lstlisting}
\end{ejemplo}

La macro \code{assert()} sirve para detener la ejecución cuando se alcanza un estado
imposible para la lógica del programa. Para usarla adecuadamente es necesario
identificar \textbf{invariantes} en el programa (condiciones que jamás deban ser
falsas). Si al evaluarse la macro resulta que su condición argumento es falsa,
\code{assert()} aborta el programa indicando nombre del archivo fuente y línea donde
estaba originalmente la llamada. Un programa en producción no debería fallar
debido a \code{assert()}.


\begin{ejemplo}
\begin{lstlisting}
#include <assert.h>
    ...
    assert(restantes >= 0);
\end{lstlisting}
\end{ejemplo}

\section{Funciones de fecha y hora}
Existen dos tipos de datos definidos en el header \code{time.h} para manejar datos de fechas.
Por un lado, se tiene el tipo \code{struct tm}, que contiene los siguientes elementos
que describen un momento en el tiempo:

\begin{lstlisting}
struct tm {
int tm_sec,    /* segundos 0..59 */
    tm_min,    /* minutos 0..59 */
    tm_hour,   /* horas 0..23 */
    tm_mday,   /* día del mes 1..31 */
    tm_mon,    /* meses desde enero 0..11 */
    tm_year,   /* años desde 1900 */
    tm_wday,   /* días desde el domingo 0..6 */
    tm_yday,   /* días desde enero 0..365 */
    tm_isdst;  /* flag de ahorro diurno de luz */
};	
\end{lstlisting}

Por otro lado, existe un segundo formato de representación interna de fechas,
\code{time_t}, que es simplemente un entero conteniendo la cantidad de segundos desde
el principio de la \textbf{era UNIX} (\quotes{\textit{the epoch}}), acaecido el 1/1/1970 a la hora 0
UTC. Este formato es el usado por la función \code{time()} que da la hora actual.
Este formato entero puede convertirse a \code{struct tm} y viceversa con funciones
definidas en esta zona de la Biblioteca Standard, como \code{mktime()} y \code{gmtime()}.

El contenido de una estructura \code{tm} se puede imprimir en una gran variedad de
formatos, con la función \code{strftime()}, que acepta una cantidad de especificaciones
al estilo de \code{printf()}. Las funciones \code{ctime()} y \code{asctime()} son más sencillas.
Devuelven una cadena conteniendo una fecha en formato normalizado (como el que
aparece en los mensajes de correo electrónico). La primera recibe un puntero a
\code{time_t}; la segunda, un puntero a \code{struct tm}.

\begin{ejemplo}
\begin{lstlisting}
time_t t;
struct tm *stm;

t = time(NULL);                                 /* recoge la hora actual */
printf("%s\n",ctime(&t));                       /* la imprime en formato standard */

char area[100];
stm = gmtime(&t);                               /* convierte t a struct tm */
strftime(area,sizeof(area),"%A %b %d %H",stm);  /* prepara string según formato del usuario */
printf("%s\n",area);                            /* lo imprime */
\end{lstlisting}
\end{ejemplo}

\section{Funciones matemáticas}
El header \code{math.h} declara las habituales funciones aritméticas avanzadas,
trigonométricas y logarítmicas. Todas ellas se presentan en \textbf{tres versiones}, que reciben y devuelven diferentes tipos de datos, y cuyos nombres están estructurados de acuerdo a dichos tipos de datos:
\begin{itemize}
	\item La versión básica recibe argumentos \code{double}, y devuelve un valor \code{double}. Por ejemplo, \code{sin(x)}.
	\item Una versión de precisión limitada, donde se agrega una \code{f} al nombre de la función, que  recibe argumentos \code{float}, y devuelve un valor \code{float}. Por ejemplo, \code{sinf(x)}.
	\item Una versión de precisión extendida, donde se agrega una \code{l} al nombre de la función, que recibe argumentos \code{long double}, y devuelve un valor \code{long double}. Por ejemplo, \code{sinl(x)}.
\end{itemize}

En el Cuadro \ref{tab:trig} se enumeran las versiones básicas. Aunque sólo se ejemplifica para $sin(x)$, todas las funciones matemáticas de la Biblioteca Standard presentan las tres versiones. Como siempre, se recomienda consultar los manuales de estas funciones.

\importante{
Para vincular un programa con la biblioteca matemática es necesario incluir la directiva de compilación \code{-lm}.
}

\tabla{trig}{Funciones matemáticas de la Biblioteca Standard.}{l|l|l}{
\hline
\code{double sin(double x);}	& Seno de $x$. & \\
\code{float sinf(float x);}	&  &  \\
\code{long double sinl(long double x);}	&  & \\
\hline
\code{double cos(double x);}	& Coseno de $x$. & \\
\hline
\code{double tan(double x);}	& Tangente de $x$. & \\
\hline
\code{double asin(double x);}	& Arco seno de $x$. & Toma valores en [$-\pi/2$, $\pi/2$].\\
\hline
\code{double acos(double x);}	& Arco coseno de $x$. & Toma valores en [$0$, $\pi$].\\
\hline
\code{double atan(double x);}	& Arco tangente de $x$. & Toma valores en  [$-\pi/2$, $\pi/2$].\\
\hline
\code{double atan2(double y,double x);}	& Arco tangente de $y/x$. & Toma valores en  [$-\pi$, $\pi$].\\
\hline
\code{double sinh(double x);} & Seno hiperbólico de $x$. \\
\hline
\code{double cosh(double x);} & Coseno hiperbólico de $x$. \\
\hline
\code{double tanh(double x);} & Tangente hiperbólica de $x$. \\
\hline
\code{double exp(double x);} & Función exponencial de base $e$. \\
\hline
\code{double log(double x);} & Logaritmo natural. \\
\hline
\code{double log10(double x);} & Logaritmo de base decimal. \\
\hline
\code{double pow(double x,double y);} & Potencia $x^y$ & Para $x=0$ debe ser $y>0$.\\
		& 			&  Si $x<0$, $y$ debe ser entero.\\
\hline
\code{double sqrt(double x);}& Raíz cuadrada de $x$.						& Debe ser $x>=0$. \\
\hline
\code{double ceil(double x);} & Menor entero no menor que $x$.						&  \\
\hline
\code{double floor(double x);} & Mayor entero no mayor que $x$.						&  \\
\hline
\code{double fabs(double x);} & Valor absoluto de $x$.						&  \\
\hline
\code{double ldexp(double x,int n);} & Devuelve $x * 2^n$.						&  \\
\hline
\code{double frexp(double x,int *exp);} & Separa a $x$ en una potencia entera 		& La potencia de 2 se almacena \\
\code{} 								&  de 2, y una mantisa en el  & en el lugar apuntado por $exp$.\\
\code{} 								&  intervalo $[\frac{1}{2}, 1].$					& Devuelve la mantisa. \\
\hline
\code{double modf(double x,int *ip);} & Separa a $x$ en parte entera 						&  \\
\code{} & y fraccionaria. 						&  \\
\hline
\code{double fmod(double x,double y);} & Residuo de punto flotante de $x/y$.						&  \\
\hline
}


\section{Funciones utilitarias}
El header \code{stdlib.h} agrupa las declaraciones de varias funciones no relacionadas
entre sí, y que sirven a varios fines. Solamente las nombramos, y encarecemos la
lectura del manual.

\begin{itemize}
	\item Funciones de conversión: las funciones \code{atoi()}, \code{atol()}, \code{atof()}, \code{strtol()},
      \code{strtod()}, \code{strtoul()}, toman cadenas representando números y generan los
      elementos de datos del tipo correspondiente.
	\item Se pueden generar números aleatorios con \code{rand()} y \code{srand()}.
	\item Aquí también se declaran las funciones de asignación de memoria como
      \code{malloc()}, \code{calloc()}, \code{realloc()}, \code{free()}.
	\item Para manejar datos en memoria con eficiencia se puede recurrir a \code{qsort()} y
      \code{bsearch()}, que ordenan una tabla y realizan búsqueda binaria en la tabla
      ordenada.
\end{itemize}

\section{Clasificación de caracteres}

El header \code{ctype.h} contiene declaraciones de macros para averiguar la
pertenencia de un carácter a determinados conjuntos (Cuadro \ref{tab:macros}). Son todas booleanas, salvo
las últimas, que devuelven ints.

\tabla{macros}{Macros de clasificación de caracteres.}{l|l}{
\hline
Macro & Devuelve \\
\hline
\code{isalnum(c)} &  \code{TRUE} cuando \code{isalpha(c)} o \code{isdigit(c)} son \code{TRUE}                                    \\
\code{isalpha(c)} &  \code{TRUE} cuando \code{isupper(c)} o \code{islower(c)} son \code{TRUE}                                     \\
\code{iscntrl(c)} &  \code{TRUE} cuando \code{c} es un carácter de control                                         \\
\code{isdigit(c)} &  \code{TRUE} cuando \code{c} es un dígito decimal                                              \\
\code{isgraph(c)} &  \code{TRUE} cuando \code{c} es un carácter imprimible y no \code{isspace(c)}                         \\
\code{islower(c)} &  \code{TRUE} cuando \code{c} es una letra minúscula                                            \\
\code{isprint(c)} &  \code{TRUE} cuando \code{c} es un carácter imprimible, \\
				   & incluyendo el caso en que \code{isspace(c)} es \code{TRUE}\\
\code{ispunct(c)} &  \code{TRUE} cuando \code{c} es imprimible pero no espacio, letra ni dígito                    \\
\code{isspace(c)} &  \code{TRUE} cuando \code{c} es espacio, fin de línea, tabulador                               \\
\code{isupper(c)} &  \code{TRUE} cuando \code{c} es letra mayúscula                                                \\
\code{isxdigit(c)} &  \code{TRUE} cuando \code{c} es dígito hexadecimal                                             \\
\code{tolower(c)} &  \code{c} en minúscula                                             \\
\code{toupper(c)} &  \code{c} en mayúscula                                             \\
}


\section{Ejercicios}
\label{sec:bstandardej}
\begin{enumerate}
	\item Utilizar la función de cantidad variable de argumentos definida más arriba
para obtener los promedios de los 2, 3, ..., n primeros elementos de un
arreglo.
	\item Construir una función de lista variable de argumentos que efectúe la
concatenación de una cantidad arbitraria de cadenas en una zona de memoria
provista por la función que llama.
	\item Construir una función de cantidad variable de argumentos que sirva para
imprimir, con un formato especificado, mensajes de debugging, conteniendo
nombres y valores de variables.
	\item Construir un programa que separe la entrada standard en palabras, usando las
macros de clasificación de caracteres. Debe considerar como delimitadores a los
caracteres espacio, tabulador, signos de puntuación, etc.
	\item Dadas dos fechas y horas del día, calcular su diferencia. Utilizar las
funciones de Biblioteca Standard para convertir a tipos de datos convenientes e imprimir la
diferencia en años, meses, días, horas, etc.
	\item Generar fechas al azar dentro de un período de tiempo dado.
\end{enumerate}

\section{Ejercicios avanzados}
\begin{enumerate}
	\item Construir un paquete de administración de archivos en formato \textbf{CSV}, escribiendo:
	\begin{itemize}
		\item Una función de inicialización que lea la cabecera del archivo, reconozca los nombres de campos y construya una lista con ellos; y prepare una tabla de registros en memoria, vacía. 
		\item Una función que lea la siguiente línea del archivo y asigne los valores a un nuevo registro en la tabla de registros en memoria.
		\item Una función que ordene alfabéticamente la tabla de registros en memoria según un campo, dado por su nombre.
		\item Una modificación a la función anterior que permita diferenciar entre campos alfanuméricos y numéricos según el nombre del campo (por ejemplo, si el nombre del campo comienza con \quotes{n}, determinar que se trata de un campo numérico), y ordene la tabla aplicando la comparación adecuada entre registros.
		\item Una función que busque una cadena en un campo alfanumérico y devuelva el índice del registro donde se la encuentre.
		\item Una función que sume los contenidos en un campo numérico.
		\item Una función que reciba un nombre de campo y un parámetro numérico o alfabético, y devuelva el índice del primer registro donde se halle un valor mayor o igual que el parámetro para ese campo, asumiendo que la tabla de registros está ordenada. 
		\item Una función que grabe el archivo con las modificaciones a los registros que hubieran ocurrido.
	\end{itemize}

Una vez probadas las funciones, escriba un programa que pueda funcionar como utilitario de línea de comandos para que el usuario administre sus archivos CSV.

Dedique algo de tiempo a considerar la mejor estructura de datos posible de acuerdo a las funciones que debe contener el paquete (por ejemplo, considere si es preferible un arreglo bidimensional definido estáticamente, un arreglo lineal definido estáticamente de punteros a espacio asignado dinámicamente, una lista asignada dinámicamente de punteros a espacio asignado dinámicamente, etc.).
\end{enumerate}

