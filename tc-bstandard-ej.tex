
\section{Ejercicios}
\label{sec:bstandardej}
\begin{enumerate}
	\item Utilizar la función de cantidad variable de argumentos definida más arriba
para obtener los promedios de los 2, 3, ..., n primeros elementos de un
arreglo.
	\item Construir una función de lista variable de argumentos que efectúe la
concatenación de una cantidad arbitraria de cadenas en una zona de memoria
provista por la función que llama.
	\item Construir una función de cantidad variable de argumentos que sirva para
imprimir, con un formato especificado, mensajes de debugging, conteniendo
nombres y valores de variables.
	\item Construir un programa que separe la entrada standard en palabras, usando las
macros de clasificación de caracteres. Debe considerar como delimitadores a los
caracteres espacio, tabulador, signos de puntuación, etc.
	\item Dadas dos fechas y horas del día, calcular su diferencia. Utilizar las
funciones de Biblioteca Standard para convertir a tipos de datos convenientes e imprimir la
diferencia en años, meses, días, horas, etc.
	\item Generar fechas al azar dentro de un período de tiempo dado.
\end{enumerate}

\section{Ejercicios avanzados}
\begin{enumerate}
	\item Construir un paquete de administración de archivos en formato \textbf{CSV}, escribiendo:
	\begin{itemize}
		\item Una función de inicialización que lea la cabecera del archivo, reconozca los nombres de campos y construya una lista con ellos; y prepare una tabla de registros en memoria, vacía. 
		\item Una función que lea la siguiente línea del archivo y asigne los valores a un nuevo registro en la tabla de registros en memoria.
		\item Una función que ordene alfabéticamente la tabla de registros en memoria según un campo, dado por su nombre.
		\item Una modificación a la función anterior que permita diferenciar entre campos alfanuméricos y numéricos según el nombre del campo (por ejemplo, si el nombre del campo comienza con \quotes{n}, determinar que se trata de un campo numérico), y ordene la tabla aplicando la comparación adecuada entre registros.
		\item Una función que busque una cadena en un campo alfanumérico y devuelva el índice del registro donde se la encuentre.
		\item Una función que sume los contenidos en un campo numérico.
		\item Una función que reciba un nombre de campo y un parámetro numérico o alfabético, y devuelva el índice del primer registro donde se halle un valor mayor o igual que el parámetro para ese campo, asumiendo que la tabla de registros está ordenada. 
		\item Una función que grabe el archivo con las modificaciones a los registros que hubieran ocurrido.
	\end{itemize}

Una vez probadas las funciones, escriba un programa que pueda funcionar como utilitario de línea de comandos para que el usuario administre sus archivos CSV.

Dedique algo de tiempo a considerar la mejor estructura de datos posible de acuerdo a las funciones que debe contener el paquete (por ejemplo, considere si es preferible un arreglo bidimensional definido estáticamente, un arreglo lineal definido estáticamente de punteros a espacio asignado dinámicamente, una lista asignada dinámicamente de punteros a espacio asignado dinámicamente, etc.).
\end{enumerate}
