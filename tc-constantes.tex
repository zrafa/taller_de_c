\chapter{Constantes}
\label{tc-constantes}
\section{Constantes numéricas}

Las constantes numéricas en un programa C pueden escribirse en varias bases. Además, la forma de escribirlas puede modificar el tamaño de los \textbf{objetos de datos} donde se almacenan.

\subsection{Constantes enteras}
\begin{itemize}
	\item \lstinline{10, -1} son constantes \textbf{decimales}.
	\item \lstinline{010, -012} son constantes \textbf{octales} por comenzar en \code{0}.
	\item \lstinline{0x10, -0x1B, 0Xbf01} son constantes \textbf{hexadecimales} por comenzar en \code{0x} o \code{0X}.
	\item \lstinline{'A'} es una constante de carácter. En una computadora que sigue la convención del código ASCII,
equivale al decimal \code{65}, hexadecimal \code{0x41}, etc.
\end{itemize}


\subsection{Constantes long} 
Una constante entera puede indicarse como \textbf{long} agregando una letra L mayúscula o minúscula: \lstinline{0L, 43l}.
Si bien numéricamente son equivalentes a 0 y a 43 int, el compilador, al encontrarlas, manejará
constantes de tamaño long (construirá objetos de datos sobre la cantidad de bits correspondientes a un
long). Esto puede ser importante en ciertas ocasiones: por ejemplo, al invocar a funciones con
argumentos formales long, usando argumentos reales que caben en un entero.
	
\subsection{Constantes unsigned}
Para hacer más claro el propósito de una constante positiva, o para forzar la promoción de una
expresión, puede notársela como \textbf{unsigned}. Esto tiene que ver con las reglas de promoción expresadas
en el capítulo 3. Constantes unsigned son, por ejemplo, \code{32u} y \code{298U}. 

\subsection{Constantes de punto flotante}
Las constantes en punto flotante se caracterizan por llevar un punto decimal o un carácter 'e' (que
indica que está en notación exponencial). Así \code{10.23, .999, 0., 1.e10, 1.e-10, 1e10} son constantes en
punto flotante. La constante \code{6.02e23} se interpreta como el número \code{6.02} multiplicado por $10^{23}$. La
constante \code{-5e-1} es igual a \code{-1/2}.


\section{Constantes string}
\label{sec:constantesstring}
El texto comprendido entre comillas dobles en un programa C es interpretado por el compilador
como una \textbf{constante string}, con propiedades similares a las de un arreglo de caracteres. 

El proceso de compilación, al identificarse una constante string, es como sigue:
\begin{itemize}
	\item El compilador reserva una zona de memoria en la imagen del programa que está construyendo, \textbf{del
tamaño del string más uno}.
	\item Se copian los caracteres entre comillas en esa zona, agregando al final \textbf{un byte conteniendo un
cero} (\lstinline{'\0'}).
	\item Se reemplaza la \textbf{referencia} a la constante string en el texto del programa por la \textbf{dirección donde
quedó almacenada}.
\end{itemize}

La cadena registrada por el compilador será almacenada al momento de ejecución en la zona del
programa correspondiente a \textbf{datos estáticos inicializados} (zona a veces llamada \quotes{bss}).
Así, una constante string \textbf{equivale a, o se evalúa a}, una dirección de memoria: la dirección donde está almacenado
su primer carácter. 

\subsection{El cero final}

El carácter \lstinline{'\0'} impuesto por el compilador al final de la secuencia de caracteres señaliza el fin de la cadena, y tiene la importante misión de funcionar como \textbf{protocolo} o \textbf{señal de terminación} para aquellas funciones de la Biblioteca Standard que manejan strings (copia de cadenas, búsqueda de caracteres, comparación de cadenas, etc.). Debido a esta representación interna, algunas veces se las ve mencionadas con el nombre de
\textbf{cadenas ASCIIZ} (caracteres ASCII seguidos de cero).

Gracias a esta representación con el \textbf{cero final}, las cadenas ASCIIZ en C \textbf{no tienen una longitud máxima}. El final de la cadena simplemente ocurre donde aparezca un carácter cero.

\importante{
Al construir o manipular cadenas ASCIIZ, es necesario cumplir con el protocolo de terminarlas con su \textbf{cero final} para
poder utilizar las funciones que trabajan con strings. De lo contrario, esas funciones \textbf{no encontrarán el final del string}.
}

¿Hay diferencias entre \lstinline{'\0'}, \lstinline{'0'} y \lstinline{"0"}? Muchas.

\begin{itemize}
	\item La primera constante, \lstinline{'\0'}, es un entero. Su valor aritmético es 0. Todos los bits del objeto de datos donde está representada son 0.
	\item La segunda, \lstinline{'0'}, es una constante de carácter. Ocupa un objeto de datos de 8 bits de tamaño. Su valor es \textbf{48 decimal} en aquellas computadoras cuyo juego de caracteres esté basado en ASCII, pero puede ser diferente en otras. 
	\item La tercera, \lstinline{"0"}, es una constante string, y se evalúa a una dirección. Es decir, en cualquier expresión donde figure, su valor aritmético es una dirección de memoria dentro del espacio de direcciones del programa. Ocupa un objeto de datos del
tamaño de una dirección (frecuentemente 16 o 32 bits), además del espacio de memoria ubicado a partir de esa dirección y ocupado por los caracteres del string. Ese espacio de memoria está ocupado por un byte igual a \lstinline{'0'} (el primer y único carácter del string), que como hemos visto equivale a 48 decimal en computadoras que adoptan ASCII, y a continuación viene un byte \lstinline{'\0'}, o sea, 0 (señal de fin del string).
\end{itemize}

\begin{ejemplo}
Si tenemos las declaraciones y asignaciones siguientes, donde las tres variables declaradas son char:
\begin{lstlisting}
char a,b,c;
a='\0';
b='0';
c="0";
\end{lstlisting}

\begin{itemize}
	\item La primera asignación es perfectamente válida y equivale a \lstinline{a=0}. 
	\item La segunda asignación también es correcta y equivale a \lstinline{b=48} en computadores basados en ASCII. 
 	\item La tercera asignación será rechazada por el compilador, generándose un error de \quotes{asignación no portable de puntero}. Los objetos a ambos lados del signo igual son de diferente naturaleza: a la izquierda tenemos algo que puede ser directamente \textbf{usado como un dato} (una constante o una variable); a la derecha, algo que, indirectamente, \textbf{referencia a un dato} (una dirección). Se dice que la variable y la constante string tienen \textbf{diferente nivel de indirección}.
\end{itemize}
\end{ejemplo}

\section{Constantes de carácter}
El gráfico muestra el resultado de asignar algunas constantes relacionadas con el problema anterior, suponiendo una arquitectura donde los enteros y las direcciones de memoria son de 16 bits. Las tres primeras asignaciones dejan en \lstinline{a} valores
aritméticos 0, 48 y 0.

Las dos últimas asignaciones dejan en \code{a} la dirección de una cadena almacenada en memoria. Las cadenas apuntadas son las que están representadas en el diagrama. 

La primera cadena contendrá el código ASCII del carácter 0 y un cero binario señalizando el fin del string. La
segunda contendrá un cero binario (expresado por la constante de carácter \lstinline{'\0'}) y un cero binario fin de string.

Las constantes de carácter son una forma expresiva y portable de especificar constantes numéricas.
Internamente, durante la compilación y ejecución del programa, el compilador las entiende como
valores numéricos sobre ocho bits. Así, es perfectamente lícito escribir expresiones como \lstinline{'A' + 1} (que
equivale a \lstinline{66}, o a \lstinline{0x42}, o a la constante de carácter \lstinline{'B'}).

Algunos caracteres especiales tienen una grafía especial:
\begin{itemize}
	\item \lstinline{\b} carácter 'backspace', ASCII 8
\item \lstinline{\t} tabulador, ASCII 9
\item \lstinline{\n} fin de línea, ASCII 10 (UNIX) o secuencia 13,10 (DOS)
\item \lstinline{\r} retorno de carro, ASCII 13
\item Una forma alternativa de escribir constantes de carácter es mediante su código ASCII:
\lstinline{'\033', '\x1B'}
Aquí representamos el carácter cuyo código ASCII es 27 decimal, en dos bases. La barra invertida
(\textit{backslash}) muestra que el contenido de las comillas simples debe ser interpretado como el código del
carácter. Si el carácter siguiente al backslash es x o X, el código está en hexadecimal; si no, está en
octal. Para representar el carácter backslash, sin su significado como modificador de secuencias de
otros caracteres, lo escribimos dos veces seguidas.
\end{itemize}


Estas constantes de carácter pueden ser también escritas respectivamente como las constantes
numéricas \code{033}, \code{27} o \code{0x1B}, ya que son aritméticamente equivalentes; pero con las comillas simples
indicamos que el programador \quotes{ve} a estas constantes como caracteres, lo que puede agregar
expresividad a un segmento de programa.

Por ejemplo, 0 es claramente una constante numérica; pero si escribimos \lstinline{'\0'} (que es numéricamente
equivalente), ponemos en evidencia que estamos pensando en el carácter cuyo código ASCII es 0. El carácter
\lstinline{'\0'} (ASCII 0) es distinto de \lstinline{'0'} (ASCII 48). La expresión de las constantes de carácter mediante
backslash y algún otro contenido se llama una \textbf{secuencia de escape}.




\subsection{Constantes de carácter en strings}
Todas estas notaciones para las constantes de carácter pueden intervenir en la escritura de constantes
string. El mecanismo de reconocimiento de constantes de caracteres dentro de strings asegura que todo el
juego de caracteres de la máquina pueda ser expresado dentro de una constante string, aun cuando no
sea imprimible o no pueda producirse con el teclado. Cuando el compilador se encuentre analizando
una constante string asignará un significado especial al carácter \textbf{barra invertida} o backslash (\code{\\}). La
aparición de un backslash permite referirse a los caracteres por su código en el sistema de la máquina
(por lo común, el ASCII).

\section{Constantes enumeradas}
Como una alternativa más legible y expresiva a la definición de constantes de preprocesador, se
pueden definir grupos de constantes reunidas por una declaración. Una declaración de \textbf{constantes
enumeradas} hace que las constantes tomen valores consecutivos de una secuencia. 

Si no se especifica el primer inicializador, vale 0. Si alguno se especifica, la inicialización de los
restantes continúa la secuencia. Las constantes de una enumeración no necesitan tener valores distintos, pero todos los nombres en las
diferentes declaraciones \lstinline{enum} de un programa deben ser diferentes.


\begin{ejemplo}
Aquí los valores de las constantes son ENE = 1, FEB = 2, MAR = 3, etc.
\begin{lstlisting}
enum meses { 
	ENE = 1, FEB, MAR, ABR, MAY, JUN, 
	JUL,AGO, SEP, OCT, NOV, DIC 
};
\end{lstlisting}
\end{ejemplo}
\begin{ejemplo}
Aquí los valores asumidos son respectivamente 0, 1, 2, 5, 6, y nuevamente 1 y 2.
\begin{lstlisting}
enum varios { ALFA, BETA, GAMMA, DELTA = 5,IOTA, PI = 1, RHO };	
\end{lstlisting}
\end{ejemplo}


\begin{preguntas}
\label{tc-constantes-preg}
\question Elija dos constantes correctamente escritas:
\correctchoice \code{0xFFU} y \code{0XABL}.
\choice \code{0ABU} y \code{010}.
\choice \code{0x10} y \code{-0dB}.
\choice Todas las anteriores.

\question Una constante de carácter correctamente escrita entre las siguientes es:
\choice \code{'0xAB'}.
\choice \lstinline{"A"}.
\correctchoice \code{'a'}.
\choice \code{265}.
\choice Todas las anteriores.

\question La constante de carácter \code{'0'} en un sistema basado en ASCII tiene el valor decimal
\choice 0.
\correctchoice 48.
\choice "0".

\question La constante de carácter \code{'\\0'} en un sistema basado en ASCII tiene el valor decimal
\correctchoice 0.
\choice 48.
\choice "0".

\question La constante string \lstinline{"A\\103"} se leerá una vez impresa como
\choice \verb+AA+.
\choice \verb+A103+.
\choice \verb+A\\103+.
\correctchoice \verb+A\103+.

\question La cadena \code{'ABC0x25'} es
\choice Una constante decimal.
\choice Una constante hexadecimal.
\choice Una constante string.
\choice Una constante de carácter.
\correctchoice Ninguna de las anteriores.
\end{preguntas}


\section{Ejercicios}
\label{tc-constantes-ej}
\begin{enumerate}
	\item Indicar si las siguientes constantes están bien formadas, y en caso afirmativo indicar su tipo y dar su
valor decimal.
		\begin{multicols}{4}
		\begin{enumerate}[label=\alph*.]
\item \lstinline{'C'} 
\item \lstinline{'0'}
\item \lstinline{'0xAB'}
\item \lstinline{70}
\item \lstinline{1A}
\item \lstinline{0xABL}
\item \lstinline{070} 
\item \lstinline{'010'} 
\item \lstinline{0xaB}
\item \lstinline{080}
\item \lstinline{0x10} 
\item \lstinline{'0xAB'}
\item \lstinline{0XFUL}
\item \lstinline{'\030'} 
\item \lstinline{-40L}
\item \lstinline{015L}
\item \lstinline{x41}
\item \lstinline{'B'}
\item \lstinline{'\xBB'} 
\item \lstinline{'AB'}
\item \lstinline{322U}
	\end{enumerate}
	\end{multicols}
\item Indicar qué caracteres componen las constantes string siguientes:
		\begin{enumerate}[label=\alph*.]
\item \lstinline{"ABC\bU\tZ"}
\item \lstinline{"\103B\x41"}
	\end{enumerate}
\item ¿Cómo se imprimirán estas constantes string?
		\begin{enumerate}[label=\alph*.]
\item \lstinline{"\0BA"}
\item \lstinline{"\\0BA"}
\item \lstinline{"BA\0CD"}
	\end{enumerate}
\item ¿Qué imprime esta sentencia? Pista: \textit{nada} no es la respuesta correcta.
\begin{lstlisting}
printf("0\r1\r2\r3\r4\r5\r6\r7\r8\r9\r");
\end{lstlisting}
\item Escribir una macro que devuelva el valor numérico de un carácter correspondiente a un dígito en base decimal.
\item Idem donde el carácter es un dígito hexadecimal entre A y F.
\end{enumerate}


