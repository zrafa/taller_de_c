
\begin{preguntas}
\label{sec:tc-preprocesador-preg}
\question El preprocesador interviene
\choice Después de la compilación del código.
\correctchoice Antes de la compilación del código.

\question El preprocesador promueve
\correctchoice La legibilidad.
\choice La redundancia.
\choice La rapidez de ejecución.
\choice Todas las anteriores.

\question El preprocesador facilita
\choice El mantenimiento.
\choice La legibilidad.
\choice La expresividad.
\correctchoice Todas las anteriores.

\question Las directivas de preprocesador
\choice Están contenidas en el lenguaje C.
\choice Son variables de texto.
\correctchoice No pertenecen al lenguaje C.
\choice Son palabras reservadas.
\choice Son funciones de C.

\question El efecto de las directivas de preprocesador abarca
\choice La función donde están declaradas.
\correctchoice La unidad de traducción.
\choice El proyecto de programación.
\choice El bloque donde están declaradas.

\question Los headers
\choice Son escritos por el usuario.
\choice Vienen con el compilador.
\correctchoice Todas las anteriores.
\choice Ninguna de las anteriores.

\question Los headers que \textbf{definen} funciones
\choice son recomendables.
\choice son imprescindibles.
\correctchoice no son recomendables.
\choice son recomendables pero no imprescindibles.

\question ¿Cuál es la directiva de preprocesador correcta si queremos definir un símbolo ALFA con valor 1?
\choice \code{#ALFA = 1}
\choice \code{#define ALFA = 1}
\correctchoice \code{#define ALFA 1}
\choice \code{#define 1 ALFA}

\question ¿Cuál es la directiva de preprocesador correcta si queremos incluir el header de Biblioteca Standard \code{stdio.h}?
\choice \code{#include stdio.h}
\choice \code{#include <stdio>}
\correctchoice \code{#include <stdio.h>}
\choice Cualquiera de las anteriores.

\question La directiva correcta para crear una macro que devuelva el doble de su argumento es
\choice \code{#DOBLE(x) 2*x}
\choice \code{#define DOBLE 2*x}
\correctchoice \code{#define DOBLE(x) 2*(x)}
\choice \code{#define DOBLE(x) 2*(x);}
\choice \code{#define DOBLE(x) 2 * (x)}

\question ¿Cuál es la directiva correcta para incluir un header llamado  \code{beta.h} situado en el directorio donde se está realizando la compilación?
\choice \code{#define <beta.h>}
\choice \code{#include <beta.h>}
\correctchoice \code{#include "beta.h"}

\question Normalmente los headers contienen
\correctchoice Declaraciones de variables y funciones.
\choice Definiciones de variables y funciones.
\choice Prototipos de directivas.
\choice Inclusión de archivos fuente.
\choice Todas las anteriores.

\question Las directivas condicionales consideran un segmento de texto
\choice sólo si la compilación resulta exitosa.
\correctchoice sólo si la condición resulta exitosa.

\question El resultado de preprocesar la macro \code{#define FUNCION(x) 3*x+1} aplicada al argumento \code{2+1} será 
\choice \code{3*3+1}
\correctchoice \code{3*2+1+1}
\choice \code{7}
\choice \code{8}

\question El problema de la expansión errónea de las macros se soluciona 
\choice Rodeando los argumentos entre signos \code{\<\>}.
\choice Rodeando los argumentos con corchetes.
\choice Poniendo la macro completa entre comillas.
\correctchoice Rodeando los argumentos con paréntesis.
\end{preguntas}
